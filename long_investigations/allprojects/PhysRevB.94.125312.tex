\section{\href{https://link.aps.org/doi/10.1103/PhysRevB.94.125312}{ Projected equations of motion approach to hybrid quantum/classical dynamics in dielectric-metal composites}}

\subsection*{Supervisor: Dr Myrta Grüning}


We introduce a hybrid method for dielectric-metal composites that describes the dynamics of the metallic system classically while retaining a quantum description of the dielectric. The time-dependent dipole moment of the classical system is mimicked by the introduction of projected equations of motion (PEOM), and the coupling between the two systems is achieved through an effective dipole-dipole interaction. To benchmark this method, we model a test system (semiconducting quantum dot-metal nanoparticle hybrid). We begin by examining the energy absorption rate, showing agreement between the PEOM method and the analytical rotating wave approximation (RWA) solution. We then investigate population inversion and show that the PEOM method provides an accurate model for the interaction under ultrashort pulse excitation where the traditional RWA breaks down.
