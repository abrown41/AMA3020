\section{\href{https://doi.org/10.3847/1538-4357/addb49}{Application of deep learning to the classification of stokes profiles: from the quiet Sun to sunspots}}

\subsection*{Supervisor: Dr Ryan Campbell}
The morphology of circular polarization profiles from solar spectropolarimetric
observations encodes information about the magnetic field strength, inclination,
and line-of-sight velocity gradients. Previous studies used manual methods or
unsupervised machine learning (ML) to classify the shapes of circular
polarization profiles. We trained a multilayer perceptron comparing
classifications with unsupervised ML. The method was tested on quiet Sun data
sets from Daniel K. Inouye Solar Telescope (DKIST), Hinode, and GREGOR, as well
as simulations of granulation and a sunspot. We achieve validation metrics
typically close to or above 90\%. We also present the first statistical analysis of quiet Sun DKIST/ViSP data using inversions and our supervised classifier. We demonstrate that classifications with unsupervised ML alone can introduce systemic errors that could compromise statistical comparisons. DKIST and Hinode classifications in the quiet Sun are similar, despite our modeling indicating spatial resolution differences should alter the shapes of circular polarization signals. Asymmetrical (symmetrical) profiles are less (more) common in GREGOR than DKIST or Hinode data, consistent with narrower response functions in the 1564.85 nm line. Single-lobed profiles are extremely rare in GREGOR data. In the sunspot simulation, the 630.25 nm line produces “double” profiles in the penumbra, likely a manifestation of magneto-optical effects in horizontal fields; these are rarer in the 1564.85 nm line. We find the 1564.85 nm line detects more reverse polarity magnetic fields in the penumbra, in contradiction to observations. We detect mixed-polarity profiles in nearly one fifth of the penumbra. Supervised ML robustly classifies solar spectropolarimetric data, enabling detailed statistical analyses of magnetic fields.
