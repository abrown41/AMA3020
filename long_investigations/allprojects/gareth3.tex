\section{\href{https://doi.org/10.1021/acs.jctc.0c01177}{Global Free-Energy Landscapes as a Smoothly Joined Collection of Local Maps}}

\subsection*{Supervisor: Dr Gareth Tribello}
Enhanced sampling techniques have become an essential tool in computational chemistry and physics, where they are applied to sample activated processes that occur on a time scale that is inaccessible to conventional simulations. Despite their popularity, it is well known that they have constraints that hinder their application to complex problems. The core issue lies in the need to describe the system using a small number of collective variables (CVs). Any slow degree of freedom that is not properly described by the chosen CVs will hinder sampling efficiency. However, the exploration of configuration space is also hampered by including variables that are not relevant for the activated process under study. This paper presents the Adaptive Topography of Landscape for Accelerated Sampling (ATLAS), a new biasing method capable of working with many CVs. The root idea of ATLAS is to apply a divide-and-conquer strategy, where the high-dimensional CVs space is divided into basins, each of which is described by an automatically determined, low-dimensional set of variables. A well-tempered metadynamics-like bias is constructed as a function of these local variables. Indicator functions associated with the basins switch on and off the local biases so that the sampling is performed on a collection of low-dimensional CV spaces that are smoothly combined to generate an effectively high-dimensional bias. The unbiased Boltzmann distribution is recovered through reweighing, making the evaluation of conformational and thermodynamic properties straightforward. The decomposition of the free-energy landscape in local basins can be updated iteratively as the simulation discovers new (meta)stable states.
