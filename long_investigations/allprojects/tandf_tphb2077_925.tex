\section{\href{https://doi.org/10.1080/13642819808206398
}{ Local heating in ballistic atomic-scale contacts
}}


\subsection*{Supervisor: Dr Tchavdar Todorov}



Recent experiments on atomic-scale metallic contacts suggest that these systems heat up with increasing electrical current. This is surprising because the size of the contacts is typically much smaller than the electron mean free path. In past work on larger nanoscale bridges, it has been proposed that isolated oscillating defects heat up in the high local current because the excited state of the current-carrying electrons is seen by the defects as having an elevated effective temperature. Here it is argued that every atom in an atomic-scale contact, and not just special defect atoms, should heat up for the same reason. The Einstein model of independent oscillators is used to obtain an effective elevated atomic temperature as a function of current in defect-free contacts. This heating is present even if the length of the contact is much smaller than the electron mean free path so that from the point of view of the electrons the contact is ballistic. Finally a rough estimate indicates that, while lattice heat conduction greatly reduces the heating effect, enough of it survives to be observable. Possible heating in defect-free ballistic contacts explains the experimentally observed heating effects without invoking defect scattering in the contacts. Furthermore it means that the experimentally observed mechanical properties of the contacts may depend on the actual applied current.
