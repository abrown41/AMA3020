\section{\href{https://dx.doi.org/10.1088/1361-6455/ac8089}{ A Dirac R-matrix calculation for the electron-impact excitation of W+
}}

\subsection*{Supervisor: Prof Cathy Ramsbottom}



Tungsten has been chosen for use as a plasma facing component in the divertor for the ITER experiment, and is currently being used on existing tokamaks such as JET. W+ plays an integral role in assessing the impurity influx from plasma facing component of tokamaks and subsequent redeposition. Together with previously calculated a neutral tungsten electron-impact dataset this study allows us to determine neighbouring spectral lines in the same wavelength window of the spectrometer, and detect if there is strong blending of overlapping lines between these two ion stages as well as providing ionisation per photon ratios for both species. The new data is to be used for tungsten erosion/redeposition diagnostics. Methods: a significantly modified version of the GRASP0 atomic structure code in conjunction with DARC (Dirac Atomic R-matrix Code) are used to calculate the Einstein A coefficients and collisional rates used to generate a synthetic W II spectrum. The W II spectrum is compared against tungsten spectral emission experiments. Results: this study is used to model the spectrum of W II, providing the predictive capability of identifying spectral lines from recent experiments. These results provide an integral part of impurity influx and redeposition determination, as the ionisation rates may be used to calculate S/XB ratios.
