\section{\href{https://iopscience.iop.org/article/10.1088/2053-1591/acf6fb}{Thermoelectric properties of cement composite analogues from first principles calculations}}
\subsection*{Supervisor: Dr Lorenzo Stella}
Buildings are responsible for a considerable fraction of the energy wasted
globally every year, and as a result, excess carbon emissions. While heat is
lost directly in colder months and climates, resulting in increased heating
loads, in hot climates cooling and ventilation is required. One avenue towards
improving the energy efficiency of buildings is to integrate thermoelectric
devices and materials within the fabric of the building to exploit the
temperature gradient between the inside and outside to do useful work.
Cement-based materials are ubiquitous in modern buildings and present an
interesting opportunity to be functionalized. We present a systematic
investigation of the electronic transport coefficients relevant to the
thermoelectric materials of the calcium silicate hydrate (C-S-H) gel analogue,
tobermorite, using Density Functional Theory calculations with the Boltzmann
transport method. The calculated values of the Seebeck coefficient are within
the typical magnitude (200--600 $\mu$V/K) indicative of a good thermoelectric
material. The tobermorite models are predicted to be intrinsically p-type
thermoelectric material because of the presence of large concentration of the
Si-O tetrahedra sites. The calculated electronic figure of merit, ZT, for the
tobermorite models have their optimal values of 0.983 at (400 K and $10^{17}$
cm$^{-3}$) for tobermorite 9 \AA, 0.985 at (400 K and $10^{17}$ cm$^{-3}$) for
tobermorite 11 \AA\ and 1.20 at (225 K and $10^{19}$ cm$^{-3}$) for tobermorite 14
\AA, respectively.