\section{\href{https://doi.org/10.3389/fams.2023.1090753
}{ Structured dynamics of the cell-cycle at multiple scales 
}}

\subsection*{Supervisor: Dr Arran Hodgkinson}


The eukaryotic cell cycle comprises 4 phases (G$_1$, S, G$_2$, and M) and is an essential component of cellular health, allowing the cell to repair damaged DNA prior to division. Facilitating this processes, p53 is activated by DNA-damage and arrests the cell cycle to allow for the repair of this damage, while mutations in the p53 gene frequently occur in cancer. As such, this process occurs on the cell-scale but affects the organism on the cell population-scale. Here, we present two models of cell cycle progression: The first of these is concerned with the cell-scale process of cell cycle progression and the temporal biochemical processes, driven by cyclins and underlying progression from one phase to the next. The second of these models concerns the cell population-scale process of cell-cycle progression and its arrest under the influence of DNA-damage and p53-activation. Both systems take advantage of structural modeling conventions to develop novels methods for describing and exploring cell-cycle dynamics on these two divergent scales. The cell-scale model represents the accumulations of cyclins across an internal cell space and demonstrates that such a formalism gives rise to a biological clock system, with definite periodicity. The cell population-scale model allows for the exploration of interactions between various regulating proteins and the DNA-damage state of the system and quantitatively demonstrates the structural dynamics which allow p53 to regulate the G$_2$- to M-phase transition and to prevent the mitosis of genetically damaged cells. A divergent periodicity and clear distribution of transition times is observed, as compared with the single-cell system. Comparison to a system with a reduced genetic repair rate shows a greater delay in cell cycle progression and an increased accumulation of cell in the G$_2$-phase, as a result of the p53 biochemical feedback mechanism.
