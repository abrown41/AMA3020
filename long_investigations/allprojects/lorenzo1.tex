\section{\href{https://doi.org/10.3390/pharmaceutics13060889}{Drug-Rich Phases Induced by Amorphous Solid Dispersion: Arbitrary or Intentional Goal in Oral
Drug Delivery?}}

\subsection*{Supervisor: Dr Lorenzo Stella}
Among many methods to mitigate the solubility limitations of drug compounds, amorphous solid dispersion (ASD) is considered to be one of the most promising strategies to enhance the dissolution and bioavailability of poorly water-soluble drugs. The enhancement of ASD in the oral absorption of drugs has been mainly attributed to the high apparent drug solubility during the dissolution. In the last decade, with the implementations of new knowledge and advanced analytical techniques, a drug-rich transient metastable phase was frequently highlighted within the supersaturation stage of the ASD dissolution. The extended drug absorption and bioavailability enhancement may be attributed to the metastability of such drug-rich phases. In this paper, we have reviewed (i) the possible theory behind the formation and stabilization of such metastable drug-rich phases, with a focus on non-classical nucleation; (ii) the additional benefits of the ASD-induced drug-rich phases for bioavailability enhancements. It is envisaged that a greater understanding of the non-classical nucleation theory and its application on the ASD design might accelerate the drug product development process in the future.
