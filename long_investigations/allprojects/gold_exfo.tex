\section{\href{https://doi.org/10.1002/adma.202419184}{Strain-Induced Decoupling Drives Gold-Assisted Exfoliation of Large-Area Monolayer 2D Crystals}}

\subsection*{Supervisor: Dr Myrta Gruning}

Gold-assisted exfoliation (GAE) is a groundbreaking mechanical exfoliation technique that produces centimeter-scale single-crystal monolayers of 2D materials. Such large, high-quality films offer unparalleled advantages over the micron-sized flakes typically produced by conventional exfoliation techniques, significantly accelerating the research and technological advancements in the field of 2D materials. Despite its wide applications, the fundamental mechanism of GAE remains poorly understood. In this study, using MoS$_2$ on Au as a model system, ultra-low frequency Raman spectroscopy is employed to elucidate how the interlayer interactions within MoS$_2$ crystals are impacted by the gold substrate. The results reveal that the coupling at the first MoS$_2$-MoS$_2$ interface between the adhered layer on the gold substrate and the adjacent layer is substantially weakened, with the binding force being reduced to nearly zero. This renders the first interface the weakest point in the system, thereby the crystal preferentially cleaves at this junction, generating large-area monolayers with sizes comparable to the parent crystal. Biaxial strain in the adhered layer, induced by the gold substrate, is identified as the driving factor for the decoupling effect. The strain-induced decoupling effect is established as the primary mechanism of GAE, which can also play a significant role in general mechanical exfoliations.
