\section{\href{https://doi.org/10.1002/bimj.202000253
}{ Robust joint modelling of longitudinal and survival data: Incorporating a time-varying degrees-of-freedom parameter
}}

\subsection*{Supervisor: Dr Lisa McFetridge}


Monitoring of individual biomarkers has the potential of explaining the hazard of survival outcomes. In practice, these measurements are intermittently observed and are known to be subject to substantial measurement error. Joint modelling of longitudinal and survival data enables us to associate intermittently measured error-prone biomarkers with risks of survival outcomes and thus plays an important role in the analysis of medical data. Most of the joint models available in the literature have been built on the Gaussian assumption. This makes them sensitive to outliers. In this work, we study a range of robust models to address this issue. Of particular interest is the common occurrence in medical data that outliers can occur with different frequencies over time, for example, in the period when patients adjust to treatment changes. Motivated by the analysis of data gathered from patients with primary biliary cirrhosis, a new model with a time-varying robustness is introduced. Through both the motivating example and a simulation study, this research not only stresses the need to account for longitudinal outliers in the analysis of medical data and in joint modelling research but also highlights the bias and inefficiency from not properly estimating the degrees-of-freedom parameter. This work presents a number of methods in addition to the time-varying robustness, and each method can be fitted using the R package robjm.
