\section{\href{https://doi.org/10.1038/s41586-022-04703-3}{Many-body theory of positron binding to polyatomic molecules}}

 \subsection*{Supervisor: Prof. Dermot Green}


Positron binding to molecules is key to extremely enhanced positron annihilation and positron-based molecular spectroscopy. Although positron binding energies have been measured for about 90 polyatomic molecules, an accurate ab initio theoretical description of positron-molecule binding has remained elusive. Of the molecules studied experimentally, ab calculations exist for only six; these calculations agree with experiments on polar molecules to at best 25 per cent accuracy and fail to predict binding in nonpolar molecules. The theoretical challenge stems from the need to accurately describe the strong many-body correlations including polarization of the electron cloud, screening of the electron-positron Coulomb interaction and the unique process of virtual-positronium formation (in which a molecular electron temporarily tunnels to the positron). Here we develop a many-body theory of positron-molecule interactions that achieves excellent agreement with experiment (to within 1 per cent in cases) and predicts binding in formamide and nucleobases. Our framework quantitatively captures the role of many-body correlations and shows their crucial effect on enhancing binding in polar molecules, enabling binding in nonpolar molecules, and increasing annihilation rates by 2 to 3 orders of magnitude. Our many-body approach can be extended to positron scattering and annihilation $\gamma$-ray spectra in molecules and condensed matter, to provide the fundamental insight and predictive capability required to improve materials science diagnostics, develop antimatter-based technologies (including positron traps, beams and positron emission tomography), and understand positrons in the Galaxy.
