\section{\href{https://link.aps.org/doi/10.1103/PhysRevMaterials.3.074003}{ Second-harmonic generation in single-layer monochalcogenides: A response from first-principles real-time simulations}}

\subsection*{Supervisor: Dr Myrta Grüning}


Second harmonic generation (SHG) of single-layer monochalcogenides, such as GaSe and InSe, has been recently reported [2D Mater. 5, 025019 (2018); J. Am. Chem. Soc. 137, 7994 (2015)] to be extremely strong with respect to bulk and multilayer forms. To clarify the origin of this strong SHG signal, we perform first-principles real-time simulations of linear and nonlinear optical properties of these two-dimensional semiconducting materials. The simulations, based on ab initio many-body theory, accurately treat the electron-hole correlation and capture excitonic effects that are deemed important to correctly predict the optical properties of such systems. We find indeed that, as observed for other 2D systems, the SHG intensity is redistributed at excitonic resonances. The obtained theoretical SHG intensity is an order of magnitude smaller than that reported at the experimental level. This result is in substantial agreement with previously published simulations which neglected the electron-hole correlation, demonstrating that many-body interactions are not at the origin of the strong SHG measured. We then show that the experimental data can be reconciled with the theoretical prediction when a single-layer model, rather than a bulk one, is used to extract the SHG coefficient from the experimental data.
