\section{\href{https://doi.org/10.1103/PhysRevB.90.115430}{Nonconservative dynamics in long atomic wires}}

\subsection*{Supervisor: Dr Daniel Dundas}



The effect of nonconservative current-induced forces on the ions in a defect-free metallic nanowire is investigated using both steady-state calculations and dynamical simulations. Nonconservative forces were found to have a major influence on the ion dynamics in these systems, but their role in increasing the kinetic energy of the ions decreases with increasing system length. The results illustrate the importance of nonconservative effects in short nanowires and the scaling of these effects with system size. The dependence on bias and ion mass can be understood with the help of a simple pen and paper model. This material highlights the benefit of simple preliminary steady-state calculations in anticipating aspects of brute-force dynamical simulations, and provides rule of thumb criteria for the design of stable quantum wires.
