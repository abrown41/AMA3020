\documentclass[11pt]{article}
\setlength{\topmargin}{-2.5cm}
\setlength{\headheight}{0in}
\setlength{\textheight}{24.5cm}
\setlength{\textwidth}{18.5cm}
\setlength{\oddsidemargin}{-0.5cm}
\setlength{\evensidemargin}{-0.5cm}

\setlength{\parindent}{0pt}
%\renewcommand{\thesection}{\arabic{section}.\hspace{-0.2em}}
\usepackage[normalem]{ulem}
\usepackage{helvet} 
\renewcommand\familydefault{\sfdefault} 
\usepackage[T1]{fontenc}
\usepackage[usenames,dvipsnames,table]{xcolor}
\usepackage{titlesec}

\titleformat{\section}
{\color{BrickRed}\normalfont\normalsize\bfseries}
{\color{BrickRed}\thesection}{1em}{}

\titleformat{\subsection}
{\color{BrickRed}\normalfont\normalsize\bf}
{\color{BrickRed}\thesection}{1em}{}

\usepackage{graphicx,soul,palatino}
\usepackage{amsmath,amsfonts,amssymb}
\usepackage{array}
\usepackage[colorlinks,urlcolor=blue]{hyperref}

\begin{document}

\title{\bf AMA3020 Long Investigations}
\author{Andrew Brown, Dan Dundas}
\date{\today}
\maketitle

\setlength{\parskip}{8pt}

Clicking on the project number (on the left hand side) will take you to the project abstract (brief description of the research). 

The title of each paper (project) is a hyperlink to the journal article. For many of these, the full text of the article is behind a paywall, so you have to be on the QUB network to access them (without paying!) 


Bear in mind that we are not expecting you to become subject experts through doing these projects, nor do we expect you to know anything about the subject before doing the project. So don't be put off the fact that many of them are physics oriented. 

As always, you will need to provide a rank-ordered list of the projects you would like to take on.

\setlength{\parskip}{-3pt}

\tableofcontents

\vspace{18pt}
\newpage
\setlength{\parskip}{0pt}
\section{\href{https://dx.doi.org/10.1088/0953-4075/40/23/013}{ Electron-impact excitation of Ar+: an improved determination of Ar impurity influx in tokamaks
}}

\subsection*{Supervisor: Prof Connor Ballance}



Electron-impact scattering data for argon and its ions continue to be of interest in studies of magnetically confined plasmas. In an earlier paper, Griffin et al (1997 J. Phys. B: At. Mol. Opt. Phys. 30 3543) employed the results of 28-term and 40-term R-matrix calculations of electron-impact excitation in Ar+ to carry out a collisional-radiative modelling study of the impurity influx of argon in tokamaks. We have now completed a 452-term R-matrix with pseudo-states (RMPS) calculation of electron-impact excitation for Ar+ in order to provide more accurate excitation data; using these improved data, we have repeated the modelling studies presented in the earlier paper. We compare our excitation data, as well as the results of the collisional radiative calculations, with those arising from the 40-term R-matrix calculation and find significant differences.

\section{\href{https://dx.doi.org/10.1088/1361-6455/ac9872}{ Core-resonance line-shape analysis of atoms undergoing strong-field ionization
}}

\subsection*{Supervisor: Dr Lynda Hutcheson}


Using attosecond transient absorption spectroscopy for time delays where the near-infrared pump and the extreme ultraviolet (XUV) probe pulses overlap, sub-cycle structures in the build-up of absorption lines in xenon ions are investigated as a function of the pump intensity during strong-field ionization. We observe a half-cycle-periodic change in the line-shape asymmetry of the ionic 4d–5p resonances. Analyzing the line shapes, we find that in particular the phase of the induced dipole emission is modified, and the magnitude of this phase modulation decreases with increasing laser intensity. We discuss the influence of ground state depletion on interfering pathways involved in XUV-assisted strong-field ionization.

\section{\href{http://dx.doi.org/10.1103/PhysRevLett.117.093201}{Extreme-Ultraviolet-Initated High-Order Harmonic Generation: Driving Inner-Valence Electrons Using Below-Threshold-Energy Extreme-Ultraviolet Light}}

\subsection*{Supervisor: Dr Andrew Brown}

We propose a novel scheme for resolving the contribution of inner- and outer-valence electrons in extreme-ultraviolet (XUV)-initiated high-harmonic generation in neon. By probing the atom with a low-energy (below the $2s$ ionization threshold) ultrashort XUV pulse, the $2p$ electron is steered away from the core, while the $2s$ electron is enabled to describe recollision trajectories. By selectively suppressing the $2p$ recollision trajectories, we can resolve the contribution of the $2s$ electron to the high-harmonic spectrum. We apply the classical trajectory model to account for the contribution of the $2s$ electron, which allows for an intuitive understanding of the process.

\section{\href{https://link.aps.org/doi/10.1103/PhysRevLett.131.203201}{ Resolving Quantum Interference Black Box through Attosecond Photoionization Spectroscopy}}

\subsection*{Supervisor: Dr Andrew Brown}



Multiphoton light-matter interactions invoke a so-called “black box” in which the experimental observations contain the quantum interference between multiple pathways. Here, we employ polarization-controlled attosecond photoelectron metrology with a partial wave manipulator to deduce the pathway interference within this quantum ‘black box” for the two-photon ionization of neon atoms. The angle-dependent and attosecond time-resolved photoelectron spectra are measured across a broad energy range. Two-photon phase shifts for each partial wave are reconstructed through the comprehensive analysis of these photoelectron spectra. We resolve the quantum interference between the degenerate p → d → p and p → s → p two-photon ionization pathways, in agreement with our theoretical simulations. Our approach thus provides an attosecond time-resolved microscope to look inside the “black box” of pathway interference in ultrafast dynamics of atoms, molecules, and condensed matter.

\section{\href{https://arxiv.org/abs/2502.19010}{Phase evolution of strong field Ionization}}

\subsection*{Supervisor: Dr Andrew Brown}
We investigate the time-dependent evolution of the dipole phase shift induced by strong-field ionisation (SFI) using attosecond transient absorption spectroscopy (ATAS) for time-delays where the pump-probe pulses overlap. We study measured and calculated time-dependent ATA spectra of the ionic $4d\rightarrow5p$ in xenon, and present the time-dependent line shape parameters in the complex plane. We attribute the complex dynamics to the contribution of three distinct processes: accumulation of ionisation, transient population and retrapping of excited states  arising from polarisation of the ground state.


%\section{\href{https://doi.org/10.1016/j.ejor.2021.12.033}{A framework for developing generalisable discrete event simulation models of hospital emergency departments}}

\subsection*{Supervisor: Dr Laura Boyle}
Discrete event simulation (DES) is routinely used to model hospital emergency departments (EDs), primarily due to its ability to represent complex patient flow processes and investigate improvement strategies. Despite this, it is clear from published studies that many DES models are not subsequently implemented in hospitals or reused for other sites. This research addresses a gap in the literature by presenting a new data-driven modelling framework ‘GE-DES’, which outlines an approach to the design and development of generalisable ED models. The nature of the framework means that it is sufficiently flexible (i) for use across multiple EDs, and (ii) for investigating hospital-specific problems through data-driven customisation. The primary aim of GE-DES is to support model reuse and implementation. The framework is demonstrated through application to a case study ED in Australia.

\section{\href{https://doi.org/10.1007/s10654-015-0112-8}{ Effect of major lifestyle risk factors, independent and jointly, on life expectancy with and without cardiovascular disease: results from the Consortium on Health and Ageing Network of Cohorts in Europe and the United States (CHANCES)}}
 \subsection*{Supervisor: Dr Karen Cairns}


Seldom have studies taken account of changes in lifestyle habits in the elderly, or investigated their impact on disease-free life expectancy (LE) and LE with cardiovascular disease (CVD). Using data on subjects aged 50+ years from three European cohorts (RCPH, ESTHER and Tromsø), we used multi-state Markov models to calculate the independent and joint effects of smoking, physical activity, obesity and alcohol consumption on LE with and without CVD. Men and women aged 50 years who have a favourable lifestyle (overweight but not obese, light/moderate drinker, non-smoker and participates in vigorous physical activity) lived between 7.4 (in Tromsø men) and 15.7 (in ESTHER women) years longer than those with an unfavourable lifestyle (overweight but not obese, light/moderate drinker, smoker and does not participate in physical activity). The greater part of the extra life years was in terms of “disease-free” years, though a healthy lifestyle was also associated with extra years lived after a CVD event. There are sizeable benefits to LE without CVD and also for survival after CVD onset when people favour a lifestyle characterized by salutary behaviours. Remaining a non-smoker yielded the greatest extra years in overall LE, when compared to the effects of routinely taking physical activity, being overweight but not obese, and drinking in moderation. The majority of the overall LE benefit is in disease free years. Therefore, it is important for policy makers and the public to know that prevention through maintaining a favourable lifestyle is “never too late”.

\section{\href{https://doi.org/10.1057/jors.2010.27
}{ Using simulation to assess cardiac first-responder schemes exhibiting stochastic and spatial complexities
}}

\subsection*{Supervisor: Dr Karen Cairns}



A Monte-Carlo simulation-based model has been constructed to assess a public health scheme involving mobile-volunteer cardiac First-Responders. The scheme being assessed aims to improve survival of Sudden-Cardiac-Arrest (SCA) patients, through reducing the time until administration of life-saving defibrillation treatment, with volunteers being paged to respond to possible SCA incidents alongside the Emergency Medical Services. The need for a model, for example, to assess the impact of the scheme in different geographical regions, was apparent upon collection of observational trial data (given it exhibited stochastic and spatial complexities). The simulation-based model developed has been validated and then used to assess the scheme's benefits in an alternative rural region (not a part of the original trial). These illustrative results conclude that the scheme may not be the most efficient use of National Health Service resources in this geographical region, thus demonstrating the importance and usefulness of simulation modelling in aiding decision making.

%\section{\href{https://link.aps.org/doi/10.1103/PhysRevResearch.4.023230}{ Quantum fluctuation theorem for dissipative processes}}

\subsection*{Supervisor: Dr Gabriele de Chiara}


We present a general quantum fluctuation theorem for the entropy production of an open quantum system coupled to multiple environments, not necessarily at equilibrium. Such a general theorem, when restricted to the weak-coupling and Markovian regime, holds for both local and global master equations, corroborating the thermodynamic consistency of local quantum master equations. The theorem is genuinely quantum, as it can be expressed in terms of conservation of a Hermitian operator, describing the dynamics of the system state operator and of the entropy change in the baths. The integral fluctuation theorem follows from the properties of such an operator. Furthermore, it is also valid when the system is described by a time-dependent Hamiltonian. As such, the quantum Jarzynski equality is a particular case of the general result presented here. Moreover, our result can be extended to nonthermal baths, as long as microreversibility is preserved. We present some numerical examples to showcase the exact results previously obtained. We finally generalize the fluctuation theorem to the case where the interaction between the system and the bath is explicitly taken into account. We show that the fluctuation theorem amounts to a relation between time-reversed dynamics of the global density matrix and a two-time correlation function along the forward dynamics involving the baths' entropy alone.

\section{\href{https://doi.org/10.1038/nnano.2008.411}{ Current-driven atomic waterwheels}}
 \subsection*{Supervisor: Dr Daniel Dundas}


A current induces forces on atoms inside the conductor that carries it. It is now possible to compute these forces from scratch, and to perform dynamical simulations of the atomic motion under current. One reason for this interest is that current can be a destructive force—it can cause atoms to migrate, resulting in damage and in the eventual failure of the conductor. But one can also ask, can current be made to do useful work on atoms? In particular, can an atomic-scale motor be driven by electrical current as it can be by other mechanisms For this to be possible, the current-induced forces on a suitable rotor must be non-conservative, so that net work can be done per revolution. Here we show that current-induced forces in atomic wires are not conservative and that they can be used, in principle, to drive an atomic-scale waterwheel.

\section{\href{https://doi.org/10.1103/PhysRevB.90.115430}{Nonconservative dynamics in long atomic wires}}

\subsection*{Supervisor: Dr Daniel Dundas}



The effect of nonconservative current-induced forces on the ions in a defect-free metallic nanowire is investigated using both steady-state calculations and dynamical simulations. Nonconservative forces were found to have a major influence on the ion dynamics in these systems, but their role in increasing the kinetic energy of the ions decreases with increasing system length. The results illustrate the importance of nonconservative effects in short nanowires and the scaling of these effects with system size. The dependence on bias and ion mass can be understood with the help of a simple pen and paper model. This material highlights the benefit of simple preliminary steady-state calculations in anticipating aspects of brute-force dynamical simulations, and provides rule of thumb criteria for the design of stable quantum wires.

\section{\href{https://link.aps.org/doi/10.1103/PhysRevA.107.L020801}{ Unexpected dipole instabilities in small molecules after ultrafast XUV irradiation}}

\subsection*{Supervisor: Dr Daniel Dundas}



We investigate the depletion of deep-lying single-electron states in the N2 dimer under the influence of very short extreme-ultraviolet (XUV) pulses. We find, first, a marked occupation inversion for a certain window of XUV energies around 50 eV, where depletion of the deepest bound valence electron state is much larger than for any other state, and second, that this occupation inversion drives a dipole instability, i.e., a spontaneous reappearance of the dipole signal long after the laser pulse is over and the initial dipole oscillations have died out. As a tool for this study, we use time-dependent density functional theory with a self-interaction correction solved on a coordinate-space grid with absorbing boundary conditions. Key observables are state-specific electron emis- sion (depletion) and photoelectron spectra (PES). The dipole instability generates additional electron emission, leading to a specific low-energy structure in PES, a signal which could be used to identify the dipole instability experimentally. The here reported dedicated depletion of a deep lying electron state by a well-tuned XUV pulse has also been found in other atoms and molecules. It provides a practicable realization of an instantaneously produced deep hole state, a situation which is often assumed ad hoc in numerous theoretical studies of energetic ultrafast processes. Moreover, the identification of the subsequent dipole instability by PES will allow one to study basic decay channels of hole states in detail.

\section{\href{https://doi.org/10.1103/PhysRevA.93.023428}{High-order-harmonic generation in benzene with linearly and circularly polarised laser pulses}}

\subsection*{Supervisor: Dr Daniel Dundas}



High-order-harmonic generation in benzene is studied using a mixed quantum-classical approach in which the electrons are described using time-dependent density functional theory while the ions move classically. The interaction with both linearly and circularly polarised infra-red ($\lambda = 800$ nm) laser pulses of duration 10 cycles (26.7 fs) is considered. The effect of allowing the ions to move is investigated as is the effect of including self-interaction corrections to the exchange-correlation functional. Our results for circularly polarised pulses are compared with previous calculations in which the ions were kept fixed and self-interaction corrections were not included while our results for linearly polarised pulses are compared with both previous calculations and experiment. We find that even for the short duration pulses considered here, the ionic motion greatly influences the harmonic spectra. While ionization and ionic displacements are greatest when linearly polarised pulses are used, the response to circularly polarised pulses is almost comparable, in agreement with previous experimental results.

\section{\href{https://doi.org/10.1038/s41586-022-04703-3}{Many-body theory of positron binding to polyatomic molecules}}

 \subsection*{Supervisor: Prof. Dermot Green}


Positron binding to molecules is key to extremely enhanced positron annihilation and positron-based molecular spectroscopy. Although positron binding energies have been measured for about 90 polyatomic molecules, an accurate ab initio theoretical description of positron-molecule binding has remained elusive. Of the molecules studied experimentally, ab calculations exist for only six; these calculations agree with experiments on polar molecules to at best 25 per cent accuracy and fail to predict binding in nonpolar molecules. The theoretical challenge stems from the need to accurately describe the strong many-body correlations including polarization of the electron cloud, screening of the electron-positron Coulomb interaction and the unique process of virtual-positronium formation (in which a molecular electron temporarily tunnels to the positron). Here we develop a many-body theory of positron-molecule interactions that achieves excellent agreement with experiment (to within 1 per cent in cases) and predicts binding in formamide and nucleobases. Our framework quantitatively captures the role of many-body correlations and shows their crucial effect on enhancing binding in polar molecules, enabling binding in nonpolar molecules, and increasing annihilation rates by 2 to 3 orders of magnitude. Our many-body approach can be extended to positron scattering and annihilation $\gamma$-ray spectra in molecules and condensed matter, to provide the fundamental insight and predictive capability required to improve materials science diagnostics, develop antimatter-based technologies (including positron traps, beams and positron emission tomography), and understand positrons in the Galaxy.

\section{\href{https://link.aps.org/doi/10.1103/PhysRevLett.123.113402}{ Positron Binding and Annihilation in Alkane Molecules}}

\subsection*{Supervisor: Dr Gleb Gribakin}



A model-potential approach has been developed to study positron interactions with molecules. Binding energies and annihilation rates are calculated for positron bound states with a range of alkane molecules, including rings and isomers. The calculated binding energies are in good agreement with experimental data, and the existence of a second bound state for $n$-alkanes (C$_n$ H$_{2n+2}$) with $n\geq12$ is predicted in accord with experiment. The annihilation rate for the ground positron bound state scales linearly with the square root of the binding energy.

\section{\href{https://link.aps.org/doi/10.1103/PhysRevLett.97.193201}{ Positron Annihilation in Molecules by Capture into Vibrational Feshbach Resonances of Infrared-Active Modes}}

\subsection*{Supervisor: Dr Gleb Gribakin}


Enhanced positron annihilation on polyatomic molecules is a long-standing and complex problem. We report the results of calculations of resonant positron annihilation on methyl halides. A free parameter of our theory is the positron binding energy. A comparison with energy-resolved annihilation rates measured for CH$_3$F, CH$_3$Cl, and CH$_3$Br [L. D. Barnes et al., Phys. Rev. A 74, 012706 (2006)] shows good agreement and yields estimates of the binding energies.

%\section{\href{https://dx.doi.org/10.1088/2516-1075/acbc5e}{ Floquet formulation of the dynamical Berry-phase approach to nonlinear optics in extended systems
}}

\subsection*{Supervisor: Dr Myrta Grüning}


We present a Floquet scheme for the ab-initio calculation of nonlinear optical properties in extended systems. This entails a reformulation of the real-time approach based on the dynamical Berry-phase polarisation (Attaccalite and Grüning 2013 Phys. Rev. B 88 1–9) and retains the advantage of being non-perturbative in the electric field. The proposed method applies to periodically-driven Hamiltonians and makes use of this symmetry to turn a time-dependent problem into a self-consistent time-independent eigenvalue problem. We implemented this Floquet scheme at the independent particle level and compared it with the real-time approach. Our reformulation reproduces real-time-calculated 2nd and 3rd order susceptibilities for a number of bulk and two-dimensional materials, while reducing the associated computational cost by one or two orders of magnitude.

%\section{\href{https://link.aps.org/doi/10.1103/PhysRevB.94.125312}{ Projected equations of motion approach to hybrid quantum/classical dynamics in dielectric-metal composites}}

\subsection*{Supervisor: Dr Myrta Grüning}


We introduce a hybrid method for dielectric-metal composites that describes the dynamics of the metallic system classically while retaining a quantum description of the dielectric. The time-dependent dipole moment of the classical system is mimicked by the introduction of projected equations of motion (PEOM), and the coupling between the two systems is achieved through an effective dipole-dipole interaction. To benchmark this method, we model a test system (semiconducting quantum dot-metal nanoparticle hybrid). We begin by examining the energy absorption rate, showing agreement between the PEOM method and the analytical rotating wave approximation (RWA) solution. We then investigate population inversion and show that the PEOM method provides an accurate model for the interaction under ultrashort pulse excitation where the traditional RWA breaks down.

%\section{\href{https://link.aps.org/doi/10.1103/PhysRevMaterials.3.074003}{ Second-harmonic generation in single-layer monochalcogenides: A response from first-principles real-time simulations}}

\subsection*{Supervisor: Dr Myrta Grüning}


Second harmonic generation (SHG) of single-layer monochalcogenides, such as GaSe and InSe, has been recently reported [2D Mater. 5, 025019 (2018); J. Am. Chem. Soc. 137, 7994 (2015)] to be extremely strong with respect to bulk and multilayer forms. To clarify the origin of this strong SHG signal, we perform first-principles real-time simulations of linear and nonlinear optical properties of these two-dimensional semiconducting materials. The simulations, based on ab initio many-body theory, accurately treat the electron-hole correlation and capture excitonic effects that are deemed important to correctly predict the optical properties of such systems. We find indeed that, as observed for other 2D systems, the SHG intensity is redistributed at excitonic resonances. The obtained theoretical SHG intensity is an order of magnitude smaller than that reported at the experimental level. This result is in substantial agreement with previously published simulations which neglected the electron-hole correlation, demonstrating that many-body interactions are not at the origin of the strong SHG measured. We then show that the experimental data can be reconciled with the theoretical prediction when a single-layer model, rather than a bulk one, is used to extract the SHG coefficient from the experimental data.

%\section{\href{https://doi.org/10.1002/adma.202419184}{Strain-Induced Decoupling Drives Gold-Assisted Exfoliation of Large-Area Monolayer 2D Crystals}}

\subsection*{Supervisor: Dr Myrta Gruning}

Gold-assisted exfoliation (GAE) is a groundbreaking mechanical exfoliation technique that produces centimeter-scale single-crystal monolayers of 2D materials. Such large, high-quality films offer unparalleled advantages over the micron-sized flakes typically produced by conventional exfoliation techniques, significantly accelerating the research and technological advancements in the field of 2D materials. Despite its wide applications, the fundamental mechanism of GAE remains poorly understood. In this study, using MoS$_2$ on Au as a model system, ultra-low frequency Raman spectroscopy is employed to elucidate how the interlayer interactions within MoS$_2$ crystals are impacted by the gold substrate. The results reveal that the coupling at the first MoS$_2$-MoS$_2$ interface between the adhered layer on the gold substrate and the adjacent layer is substantially weakened, with the binding force being reduced to nearly zero. This renders the first interface the weakest point in the system, thereby the crystal preferentially cleaves at this junction, generating large-area monolayers with sizes comparable to the parent crystal. Biaxial strain in the adhered layer, induced by the gold substrate, is identified as the driving factor for the decoupling effect. The strain-induced decoupling effect is established as the primary mechanism of GAE, which can also play a significant role in general mechanical exfoliations.

\section{\href{https://doi.org/10.3389/fams.2023.1090753
}{ Structured dynamics of the cell-cycle at multiple scales 
}}

\subsection*{Supervisor: Dr Arran Hodgkinson}


The eukaryotic cell cycle comprises 4 phases (G$_1$, S, G$_2$, and M) and is an essential component of cellular health, allowing the cell to repair damaged DNA prior to division. Facilitating this processes, p53 is activated by DNA-damage and arrests the cell cycle to allow for the repair of this damage, while mutations in the p53 gene frequently occur in cancer. As such, this process occurs on the cell-scale but affects the organism on the cell population-scale. Here, we present two models of cell cycle progression: The first of these is concerned with the cell-scale process of cell cycle progression and the temporal biochemical processes, driven by cyclins and underlying progression from one phase to the next. The second of these models concerns the cell population-scale process of cell-cycle progression and its arrest under the influence of DNA-damage and p53-activation. Both systems take advantage of structural modeling conventions to develop novels methods for describing and exploring cell-cycle dynamics on these two divergent scales. The cell-scale model represents the accumulations of cyclins across an internal cell space and demonstrates that such a formalism gives rise to a biological clock system, with definite periodicity. The cell population-scale model allows for the exploration of interactions between various regulating proteins and the DNA-damage state of the system and quantitatively demonstrates the structural dynamics which allow p53 to regulate the G$_2$- to M-phase transition and to prevent the mitosis of genetically damaged cells. A divergent periodicity and clear distribution of transition times is observed, as compared with the single-cell system. Comparison to a system with a reduced genetic repair rate shows a greater delay in cell cycle progression and an increased accumulation of cell in the G$_2$-phase, as a result of the p53 biochemical feedback mechanism.

\section{\href{https://doi.org/10.3389/fonc.2022.857572}{Computational Model of Heterogeneity in Melanoma: Designing Therapies and Predicting Outcomes}}

\subsection*{Supervisor: Dr Arran Hodgkinson}


Cutaneous melanoma is a highly invasive tumor and, despite the development of recent therapies, most patients with advanced metastatic melanoma have a poor clinical outcome. The most frequent mutations in melanoma affect the BRAF oncogene, a protein kinase of the MAPK signaling pathway. Therapies targeting both BRAF and MEK are effective for only 50\% of patients and, almost systematically, generate drug resistance. Genetic and non-genetic mechanisms associated with the strong heterogeneity and plasticity of melanoma cells have been suggested to favor drug resistance but are still poorly understood. Recently, we have introduced a novel mathematical formalism allowing the representation of the relation between tumor heterogeneity and drug resistance and proposed several models for the development of resistance of melanoma treated with BRAF/MEK inhibitors. In this paper, we further investigate this relationship by using a new computational model that copes with multiple cell states identified by single cell mRNA sequencing data in melanoma treated with BRAF/MEK inhibitors. We use this model to predict the outcome of different therapeutic strategies. The reference therapy, referred to as “continuous” consists in applying one or several drugs without disruption. In “combination therapy”, several drugs are used sequentially. In “adaptive therapy” drug application is interrupted when the tumor size is below a lower threshold and resumed when the size goes over an upper threshold. We show that, counter-intuitively, the optimal protocol in combination therapy of BRAF/MEK inhibitors with a hypothetical drug targeting cell states that develop later during the tumor response to kinase inhibitors, is to treat first with this hypothetical drug. Also, even though there is little difference in the timing of emergence of the resistance between continuous and adaptive therapies, the spatial distribution of the different melanoma subpopulations is more zonated in the case of adaptive therapy.

%\section{\href{https://doi.org/10.1007/s10884-017-9599-x}{ Rapidly and Slowly Oscillating Periodic Solutions of a Delayed Van der Pol Oscillator}}
 \subsection*{Supervisor: Dr Gabor Kiss}


In this paper, we introduce a method to prove existence of several rapidly and slowly oscillating periodic solutions of a delayed Van der Pol oscillator. The proof is a combination of pen and paper analytic estimates, the contraction mapping theorem and a computer program using interval arithmetic. Using this approach we extend some existence results obtained by Nussbaum (Ann Mat Pura Appl 4(101):263–306, 1974).

%\section{\href{https://doi.org/10.1016/j.jde.2011.11.020}{ Computational fixed-point theory for differential delay equations with multiple time lags}}

\subsection*{Supervisor: Dr Gabor Kiss}



We introduce a general computational fixed-point method to prove existence of periodic solutions of differential delay equations with multiple time lags. The idea of such a method is to compute numerical approximations of periodic solutions using Newton's method applied on a finite dimensional projection, to derive a set of analytic estimates to bound the truncation error term and finally to use this explicit information to verify computationally the hypotheses of a contraction mapping theorem in a given Banach space. The fixed point so obtained gives us the desired periodic solution. We provide two applications. The first one is a proof of coexistence of three periodic solutions for a given delay equation with two time lags, and the second one provides rigorous computations of several nontrivial periodic solutions for a delay equation with three time lags.

%\section{\href{https://doi.org/10.1158/1078-0432.CCR-22-1102}{ Biological Misinterpretation of Transcriptional Signatures in Tumor Samples Can Unknowingly Undermine Mechanistic Understanding and Faithful Alignment with Preclinical Data
}}

\subsection*{Supervisor: Dr Felicity Lamrock}



Precise mechanism-based gene expression signatures (GES) have been developed in appropriate in vitro and in vivo model systems, to identify important cancer-related signaling processes. However, some GESs originally developed to represent specific disease processes, primarily with an epithelial cell focus, are being applied to heterogeneous tumor samples where the expression of the genes in the signature may no longer be epithelial-specific. Therefore, unknowingly, even small changes in tumor stroma percentage can directly influence GESs, undermining the intended mechanistic signaling.Using colorectal cancer as an exemplar, we deployed numerous orthogonal profiling methodologies, including laser capture microdissection, flow cytometry, bulk and multiregional biopsy clinical samples, single-cell RNA sequencing and finally spatial transcriptomics, to perform a comprehensive assessment of the potential for the most widely used GESs to be influenced, or confounded, by stromal content in tumor tissue. To complement this work, we generated a freely-available resource, ConfoundR; https://confoundr.qub.ac.uk/, that enables users to test the extent of stromal influence on an unlimited number of the genes/signatures simultaneously across colorectal, breast, pancreatic, ovarian and prostate cancer datasets.Findings presented here demonstrate the clear potential for misinterpretation of the meaning of GESs, due to widespread stromal influences, which in-turn can undermine faithful alignment between clinical samples and preclinical data/models, particularly cell lines and organoids, or tumor models not fully recapitulating the stromal and immune microenvironment.Efforts to faithfully align preclinical models of disease using phenotypically-designed GESs must ensure that the signatures themselves remain representative of the same biology when applied to clinical samples.

\section{\href{https://doi.org/10.1016/j.ins.2011.02.001}{ Concordance and consensus}}

\subsection*{Supervisor: Dr Zhiwei Lin}



This paper deals with the measurement of concordance and the construction of consensus in preference data, either in the form of preference rankings or in the form of response distributions with Likert-items. We propose a set of axioms of concordance in preference orderings and a new class of concordance measures. The measures outperform classic measures like Kendall's $\tau$ and W and Spearman's $\rho$ in sensitivity and apply to large sets of orderings instead of just to pairs of orderings. For sets of N orderings of n items, we present very efficient and flexible algorithms that have a time complexity of only O(Nn2). Remarkably, the algorithms also allow for fast calculation of all longest common subsequences of the full set of orderings. We experimentally demonstrate the performance of the algorithms. A new and simple measure for assessing concordance on Likert-items is proposed.

%\section{\href{https://doi.org/10.1016/S1473-3099(20)30474-6}{ Validating clinical practice guidelines for the management of children with non-blanching rashes in the UK (PiC): a prospective, multicentre cohort study
}}

\subsection*{Supervisor: Dr Lisa McFetridge}


No previous studies have validated current clinical practice guidelines for the management of non-blanching rashes in children who have received meningococcal B and C vaccinations. The aim of this study was to evaluate the performance of existing clinical practice guidelines in the diagnosis of invasive meningococcal disease in children presenting with a fever and non-blanching rash in the UK.

%\section{\href{https://doi.org/10.1002/bimj.202000253
}{ Robust joint modelling of longitudinal and survival data: Incorporating a time-varying degrees-of-freedom parameter
}}

\subsection*{Supervisor: Dr Lisa McFetridge}


Monitoring of individual biomarkers has the potential of explaining the hazard of survival outcomes. In practice, these measurements are intermittently observed and are known to be subject to substantial measurement error. Joint modelling of longitudinal and survival data enables us to associate intermittently measured error-prone biomarkers with risks of survival outcomes and thus plays an important role in the analysis of medical data. Most of the joint models available in the literature have been built on the Gaussian assumption. This makes them sensitive to outliers. In this work, we study a range of robust models to address this issue. Of particular interest is the common occurrence in medical data that outliers can occur with different frequencies over time, for example, in the period when patients adjust to treatment changes. Motivated by the analysis of data gathered from patients with primary biliary cirrhosis, a new model with a time-varying robustness is introduced. Through both the motivating example and a simulation study, this research not only stresses the need to account for longitudinal outliers in the analysis of medical data and in joint modelling research but also highlights the bias and inefficiency from not properly estimating the degrees-of-freedom parameter. This work presents a number of methods in addition to the time-varying robustness, and each method can be fitted using the R package robjm.

\section{\href{https://doi.org/10.1177/15385744221149585
}{ Open Surgery for Abdominal Aortic Aneurysm: 980 Consecutive Patient Outcomes from a High-Volume Centre in the United Kingdom
}}

\subsection*{Supervisor: Dr Hannah Mitchell}



Controversy persists regarding the optimal treatment for large abdominal aortic aneurysm (AAA), highlighted by the publication of the National Institute for Health and Care Excellence (NICE) guideline (NG156) on March 2020. The pendulum of opinion swings once more from endovascular to open surgical treatment. We report our experience over the last 15 years in treating consecutive AAA by open surgery.MethodsA retrospective review of a prospectively collected vascular database of all patients undergoing infra-renal open abdominal aortic aneurysm repair (OR) repair from 2004 to 2019 at the largest aneurysm centre in the United Kingdom. OR for elective and emergency (ruptured and symptomatic) outcomes included early morbidity and 30-day mortality, and long-term survival.Results1017 patients underwent OR between 2004-2019, on application of our inclusion-criteria 994 patients formed our cohort for analysis (81.2\% male) with a mean age 73.6 ± 7.8 years treated by OR for AAA. In that group 672 were elective and 308 were emergency (for ruptured or symptomatic). Overall 30 day mortality was 11.3\%, elective 30 day mortality was 2.5\%, and emergency 30 day mortality was 29.9\%. 30 day re-intervention rate was 9.5\%, (elective 7.0\%, emergency 15.9\%). Survival at 1000 days for elective repair was 72 v 46.7\% for emergency and at 2000 days was 43.4\% for elective v 25\% for emergency.ConclusionOur data confirm that open surgery for AAA can be performed in large volume centres quite safely. Elective and Emergency surgery does affect early 30 day mortality but does not influence long-term outcome.

%\input{IOPEXPORT_RIS-3}
%\section{\href{https://link.aps.org/doi/10.1103/PhysRevA.98.020101}{ Experimental signature of quantum Darwinism in photonic cluster states}}

\subsection*{Supervisor: Prof Mauro Paternostro}


We report on an experimental assessment of the emergence of Quantum Darwinism (QD) from engineered open-system dynamics. We use a photonic hyperentangled source of graph states to address the effects that correlations among the elements of a multiparty environment have on the establishment of objective reality ensuing the quantum-to-classical transition. Besides embodying one of the first experimental efforts toward the characterization of QD, our work illustrates the nontrivial consequences that multipartite entanglement and, in turn, the possibility of having environment-to-system back-action have on the features of the QD framework.

%\section{\href{https://dx.doi.org/10.1088/1361-6455/ac8089}{ A Dirac R-matrix calculation for the electron-impact excitation of W+
}}

\subsection*{Supervisor: Prof Cathy Ramsbottom}



Tungsten has been chosen for use as a plasma facing component in the divertor for the ITER experiment, and is currently being used on existing tokamaks such as JET. W+ plays an integral role in assessing the impurity influx from plasma facing component of tokamaks and subsequent redeposition. Together with previously calculated a neutral tungsten electron-impact dataset this study allows us to determine neighbouring spectral lines in the same wavelength window of the spectrometer, and detect if there is strong blending of overlapping lines between these two ion stages as well as providing ionisation per photon ratios for both species. The new data is to be used for tungsten erosion/redeposition diagnostics. Methods: a significantly modified version of the GRASP0 atomic structure code in conjunction with DARC (Dirac Atomic R-matrix Code) are used to calculate the Einstein A coefficients and collisional rates used to generate a synthetic W II spectrum. The W II spectrum is compared against tungsten spectral emission experiments. Results: this study is used to model the spectrum of W II, providing the predictive capability of identifying spectral lines from recent experiments. These results provide an integral part of impurity influx and redeposition determination, as the ionisation rates may be used to calculate S/XB ratios.

%\section{\href{https://doi.org/10.1093/mnras/sty3198}{ Towards converged electron-impact excitation calculations of low-lying transitions in Fe ii}}

\subsection*{Supervisor: Prof Cathy Ramsbottom}


Absorption and emission lines of the iron-peak species Fe ii are prominent in the infrared, optical, and ultraviolet spectra of a myriad of astrophysical sources, requiring extensive and highly reliable sets of atomic structure and collisional data for an accurate quantitative analysis. However, comparisons among existing calculations reveal large discrepancies in the effective collision strengths, often up to factors of three, highlighting the need for further steps towards new converged calculations. Here we report a new 20 configuration, 6069 level atomic structure model, calculated using the multiconfigurational Dirac-Fock method. Collision strengths and effective collision strengths are presented, for a wide range of temperatures of astrophysical relevance, from substantial 262 level and 716 level Dirac R-matrix calculations, plus a 716 level Breit-Pauli R-matrix calculation. Convergence of the scattering calculations is discussed, and results are critically compared with existing data in the literature, providing us with error estimates for our data. As a consequence, we assign an uncertainty of ±15 per cent to relevant forbidden and allowed transitions encompassed within a 50 level subset of the 716 level Dirac R-matrix data set. To illustrate the implications of our new data sets for the analysis of astronomical observations of Fe ii, they are incorporated into the cloudy modelling code, sample Fe ii spectra are generated and compared.

\section{\href{https://pubs.acs.org/doi/10.1021/acs.iecr.8b05709}{Development and Optimization of an Immobilized Photocatalytic System within a Stacked Frame Photoreactor (SFPR) Using Light Distribution and Fluid Mixing Simulation Coupled with Experimental Validation}}

\subsection*{Supervisor: Dr Lorenzo Stella}
Recently, photocatalytic reactors have been designed with a view toward overcoming mass transfer limitations especially in systems with immobilized catalysts. This paper reports the design of a titanium “bladed” propeller with TiO2 immobilized on the blades. To evaluate the propeller efficiency, modeling using COMSOL Multiphysics was validated experimentally using coumarin as a probe molecule allowing for OH radical quantification. Modeling of light distribution and catalyst irradiance at varying irradiation distance was performed using ray optics, which, alongside experimental work, showed that irradiation at 4 and 5 cm from the propeller yielded the highest irradiance (29.3 and 22.1~mW/cm$^2$) and OH radical concentrations (5.38 and 5.56 $\mu$M, respectively). Propeller rotation was modeled and compared against experimental data to assess mass transfer limitations at varying rotation speeds. This showed that 300 rpm provided the highest rate of coumarin degradation (0.32 $\mu$M/min) despite the model showing higher fluid velocities at 400~rpm.

%\section{\href{https://dx.doi.org/10.1088/0953-8984/5/15/010}{ Elastic quantum transport through small structures
}}

\subsection*{Supervisor: Dr Tchavdar Todorov}



The authors' develop a general formulation of the problem of elastic transport between two semi-infinite systems, connected by a system of finite size, and derive expressions for the current in and the differential conductance of such a circuit in the limit of zero interactions between the carriers. These expressions are exact in the applied voltage, the coupling of the components of the circuit, and the temperature of the circuit. They then apply their results in a tight-binding approximation to three specific cases: the one-atom contact, the finite, disordered one-dimensional chain, and the generalized stacking fault.

%\section{\href{https://doi.org/10.1080/13642819808206398
}{ Local heating in ballistic atomic-scale contacts
}}


\subsection*{Supervisor: Dr Tchavdar Todorov}



Recent experiments on atomic-scale metallic contacts suggest that these systems heat up with increasing electrical current. This is surprising because the size of the contacts is typically much smaller than the electron mean free path. In past work on larger nanoscale bridges, it has been proposed that isolated oscillating defects heat up in the high local current because the excited state of the current-carrying electrons is seen by the defects as having an elevated effective temperature. Here it is argued that every atom in an atomic-scale contact, and not just special defect atoms, should heat up for the same reason. The Einstein model of independent oscillators is used to obtain an effective elevated atomic temperature as a function of current in defect-free contacts. This heating is present even if the length of the contact is much smaller than the electron mean free path so that from the point of view of the electrons the contact is ballistic. Finally a rough estimate indicates that, while lattice heat conduction greatly reduces the heating effect, enough of it survives to be observable. Possible heating in defect-free ballistic contacts explains the experimentally observed heating effects without invoking defect scattering in the contacts. Furthermore it means that the experimentally observed mechanical properties of the contacts may depend on the actual applied current.

\section{\href{https://doi.org/10.1103/PhysRevB.110.024101}{Modeling the ferroelectric phase transition in barium titanate with DFT accuracy and converged sampling}}

\subsection*{Supervisor: Dr Gareth Tribello}
The accurate description of the structural and thermodynamic properties of ferroelectrics has been one of the most remarkable achievements of density functional theory (DFT). However, running large simulation cells with DFT is computationally demanding, while simulations of small cells are often plagued with nonphysical effects that are a consequence of the system's finite size. To avoid these finite-size effects one is thus often forced to use empirical models that describe the physics of the material in terms of effective interaction terms, that are fitted using the results from DFT. In this study we use a machine-learning (ML) potential trained on DFT, in combination with accelerated sampling techniques, to converge the thermodynamic properties of barium titanate (BTO) with first-principles accuracy and a full atomistic description. Our results indicate that the predicted Curie temperature depends strongly on the choice of DFT functional and system size, because of emergent long-range directional correlations in the local dipole fluctuations. Our findings demonstrate how the combination of ML models and traditional bottom-up modeling allow one to investigate emergent phenomena with the accuracy of first-principles calculations over the large size and time scales afforded by empirical models.

\section{\href{https://doi.org/10.1063/1.5134461
}{ Classical nucleation theory predicts the shape of the nucleus in homogeneous solidification
}}

\subsection*{Supervisor: Dr Gareth Tribello}



Macroscopic models of nucleation provide powerful tools for understanding activated phase transition processes. These models do not provide atomistic insights and can thus sometimes lack material-specific descriptions. Here, we provide a comprehensive framework for constructing a continuum picture from an atomistic simulation of homogeneous nucleation. We use this framework to determine the equilibrium shape of the solid nucleus that forms inside bulk liquid for a Lennard-Jones potential. From this shape, we then extract the anisotropy of the solid-liquid interfacial free energy, by performing a reverse Wulff construction in the space of spherical harmonic expansions. We find that the shape of the nucleus is nearly spherical and that its anisotropy can be perfectly described using classical models.

\section{\href{https://doi.org/10.1021/acs.jpclett.1c02574
}{ Efficient Quantum Vibrational Spectroscopy of Water with High-Order Path Integrals: From Bulk to Interfaces
}}

\subsection*{Supervisor: Dr David Wilkins}



Vibrational spectroscopy is key in probing the interplay between the structure and dynamics of aqueous systems. To map different regions of experimental spectra to the microscopic structure of a system, it is important to combine them with first-principles atomistic simulations that incorporate the quantum nature of nuclei. Here we show that the large cost of calculating the quantum vibrational spectra of aqueous systems can be dramatically reduced compared with standard path integral methods by using approximate quantum dynamics based on high-order path integrals. Together with state-of-the-art machine-learned electronic properties, our approach gives an excellent description not only of the infrared and Raman spectra of bulk water but also of the 2D correlation and the more challenging sum-frequency generation spectra of the water–air interface. This paves the way for understanding complex interfaces such as water encapsulated between or in contact with hydrophobic and hydrophilic materials through robust and inexpensive surface-sensitive and multidimensional spectra with first-principles accuracy.

\section{\href{https://doi.org/10.1021/acs.jpclett.2c03896
}{ Competing Nuclear Quantum Effects and Hydrogen-Bond Jumps in Hydrated Kaolinite}}

\subsection*{Supervisor: Dr David Wilkins}



Recent work has shown that the dynamics of hydrogen bonds in pure clays are affected by nuclear quantum fluctuations, with different effects for the hydrogen bonds holding different layers of the clay together and for those within the same layer. At the clay–water interface there is an even wider range of types of hydrogen bond, suggesting that the quantum effects may be yet more varied. We apply classical and thermostated ring polymer molecular dynamics simulations to show that nuclear quantum effects accelerate hydrogen-bond dynamics to varying degrees. By interpreting the results in terms of the extended jump model of hydrogen-bond switching, we can understand the origins of these effects in terms of changes in the quantum kinetic energy of hydrogen atoms during an exchange. We also show that the extended jump mechanism is applicable not only to the hydrogen bonds involving water, but also those internal to the clay.

%\section{\href{https://doi.org/10.48550/arXiv.2308.05676}{Modelling non-local cell-cell adhesion: a multiscale approach}}

\subsection*{Supervisor: Dr Anna Zhigun}


Cell-cell adhesion plays a vital role in the development and maintenance of multicellular organisms. One of its functions is regulation of cell migration, such as occurs, e.g. during embryogenesis or in cancer. In this work, we develop a versatile multiscale approach to modelling a moving self-adhesive cell population that combines a careful microscopic description of a deterministic adhesion-driven motion component with an efficient mesoscopic representation of a stochastic velocity-jump process. This approach gives rise to mesoscopic models in the form of kinetic transport equations featuring multiple non-localities. Subsequent parabolic and hyperbolic scalings produce general classes of equations with non-local adhesion and myopic diffusion, a special case being the classical macroscopic model proposed in [4]. Our simulations show how the combination of the two motion effects can unfold.


\end{document}
