\section{\href{https://doi.org/10.1177/17479541251413077}{Ballistic power: GPS-derived measurements suitable for the real-time monitoring of neuromuscular fatigue during repeated sprints}}

\subsection*{Supervisor: Dr Sam Grant}
Power-force-velocity ($PFv$) measurements have traditionally been employed to 
characterize the limits of an athlete’s neuromuscular system. However, in more 
recent years, there has been growing interest in whether $PFv$ metrics can also 
be used to define and assess training outcomes, e.g., as an indicator of 
performance and/or neuromuscular fatigue. Here we assess performance in a 
repeated sprint ability ($RSA$) test using traditional $PFv$ measurements, 
alongside two novel metrics: ballistic power ($P_B$) and the $Fv$-offset, with 
the changes observed attributed to neuromuscular fatigue. Twenty-four physically 
active males (age = 26.08 $\pm$ 6.84 years) undertook the $RSA$ test, consisting 
of 3 sets of $5\times 50$ m maximal sprints, with each participant wearing a 
STATSports Apex Pro series unit to track their motion. Mixed-effect models 
show how power-related metrics have the largest reduction due to fatigue 
compared to their force and velocity counterparts, with PB showing the largest 
decrease of $\sim 31.8\% (-40.2\le P_B (\%$ change) $\le -23.4, 95\% $CI, 
Cohen’s $d=1.60$) across the $RSA$ test performed. We demonstrate that ballistic 
power, $P_B$, is the most reactive metric to athlete fatigue amongst the 
considered values. The described methods provide a promising, feasible tool 
for field-based monitoring of fatigue using routine GPS data.
