\section{\href{https://doi.org/10.1103/PhysRevA.48.546}{Calculation of the scattering length in atomic collisions using the semiclassical approximation}}

\subsection*{Supervisor: Dr Gleb Gribakin}
A simple analytical formula, $a = \bar{a}[1 - \tan(\pi/(n-2))\tan\{\Phi - [\pi/2(n-2)]\}]$, is obtained for the scattering length in atomic collisions. Here $\bar{a} = \cos[\pi/(n-2)]\{\sqrt{2M\alpha}/[\hbar(n-2)]\}^{2/(n-2)}[\Gamma(n-3)/(n-2)]/[T(n-1)/(n-2)]$ is the mean scattering length determined by the asymptotic behavior of the potential $U(r) \sim -\alpha/r^n$, ($n=6$ for atom--atom scattering or $n=4$ for ion--atom scattering), $M$ is the reduced mass of the atoms, and $\Phi$ is the semiclassical phase calculated at zero energy from the classical turning point to infinity. The value of $\bar{a}$, the average scattering length, also determines the slope of the $s$-wave phase shifts beyond the near-threshold region. The formula is applicable to the collisions of atoms cooled down in traps, where the scattering length determines the character of the atom--atom interaction. Our calculation shows that repulsion between atoms ($a>0$) is more likely than attraction with a ``probability'' of 75\%. For the Cs--Cs scattering in the $^{3}\Sigma^{u}$ state, $\bar{a} = 95.5\,a_{B}$ has been obtained, where $a_{B}$ is the Bohr radius. The comparison of the calculated cross-section energy dependence with the experimental data gives evidence for a positive value for the Cs--Cs scattering length, which makes cesium Bose gas stable.
