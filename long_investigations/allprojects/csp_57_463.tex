\section{\href{https://doi.org/10.1177/15385744221149585
}{ Open Surgery for Abdominal Aortic Aneurysm: 980 Consecutive Patient Outcomes from a High-Volume Centre in the United Kingdom
}}

\subsection*{Supervisor: Dr Hannah Mitchell}



Controversy persists regarding the optimal treatment for large abdominal aortic aneurysm (AAA), highlighted by the publication of the National Institute for Health and Care Excellence (NICE) guideline (NG156) on March 2020. The pendulum of opinion swings once more from endovascular to open surgical treatment. We report our experience over the last 15 years in treating consecutive AAA by open surgery.MethodsA retrospective review of a prospectively collected vascular database of all patients undergoing infra-renal open abdominal aortic aneurysm repair (OR) repair from 2004 to 2019 at the largest aneurysm centre in the United Kingdom. OR for elective and emergency (ruptured and symptomatic) outcomes included early morbidity and 30-day mortality, and long-term survival.Results1017 patients underwent OR between 2004-2019, on application of our inclusion-criteria 994 patients formed our cohort for analysis (81.2\% male) with a mean age 73.6 ± 7.8 years treated by OR for AAA. In that group 672 were elective and 308 were emergency (for ruptured or symptomatic). Overall 30 day mortality was 11.3\%, elective 30 day mortality was 2.5\%, and emergency 30 day mortality was 29.9\%. 30 day re-intervention rate was 9.5\%, (elective 7.0\%, emergency 15.9\%). Survival at 1000 days for elective repair was 72 v 46.7\% for emergency and at 2000 days was 43.4\% for elective v 25\% for emergency.ConclusionOur data confirm that open surgery for AAA can be performed in large volume centres quite safely. Elective and Emergency surgery does affect early 30 day mortality but does not influence long-term outcome.
