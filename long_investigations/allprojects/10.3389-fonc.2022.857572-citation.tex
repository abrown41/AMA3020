\section{\href{https://doi.org/10.3389/fonc.2022.857572}{Computational Model of Heterogeneity in Melanoma: Designing Therapies and Predicting Outcomes}}

\subsection*{Supervisor: Dr Arran Hodgkinson}


Cutaneous melanoma is a highly invasive tumor and, despite the development of recent therapies, most patients with advanced metastatic melanoma have a poor clinical outcome. The most frequent mutations in melanoma affect the BRAF oncogene, a protein kinase of the MAPK signaling pathway. Therapies targeting both BRAF and MEK are effective for only 50\% of patients and, almost systematically, generate drug resistance. Genetic and non-genetic mechanisms associated with the strong heterogeneity and plasticity of melanoma cells have been suggested to favor drug resistance but are still poorly understood. Recently, we have introduced a novel mathematical formalism allowing the representation of the relation between tumor heterogeneity and drug resistance and proposed several models for the development of resistance of melanoma treated with BRAF/MEK inhibitors. In this paper, we further investigate this relationship by using a new computational model that copes with multiple cell states identified by single cell mRNA sequencing data in melanoma treated with BRAF/MEK inhibitors. We use this model to predict the outcome of different therapeutic strategies. The reference therapy, referred to as “continuous” consists in applying one or several drugs without disruption. In “combination therapy”, several drugs are used sequentially. In “adaptive therapy” drug application is interrupted when the tumor size is below a lower threshold and resumed when the size goes over an upper threshold. We show that, counter-intuitively, the optimal protocol in combination therapy of BRAF/MEK inhibitors with a hypothetical drug targeting cell states that develop later during the tumor response to kinase inhibitors, is to treat first with this hypothetical drug. Also, even though there is little difference in the timing of emergence of the resistance between continuous and adaptive therapies, the spatial distribution of the different melanoma subpopulations is more zonated in the case of adaptive therapy.
