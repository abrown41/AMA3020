\section{\href{https://doi.org/10.1021/acs.jpclett.2c03896
}{ Competing Nuclear Quantum Effects and Hydrogen-Bond Jumps in Hydrated Kaolinite}}

\subsection*{Supervisor: Dr David Wilkins}



Recent work has shown that the dynamics of hydrogen bonds in pure clays are affected by nuclear quantum fluctuations, with different effects for the hydrogen bonds holding different layers of the clay together and for those within the same layer. At the clay–water interface there is an even wider range of types of hydrogen bond, suggesting that the quantum effects may be yet more varied. We apply classical and thermostated ring polymer molecular dynamics simulations to show that nuclear quantum effects accelerate hydrogen-bond dynamics to varying degrees. By interpreting the results in terms of the extended jump model of hydrogen-bond switching, we can understand the origins of these effects in terms of changes in the quantum kinetic energy of hydrogen atoms during an exchange. We also show that the extended jump mechanism is applicable not only to the hydrogen bonds involving water, but also those internal to the clay.
