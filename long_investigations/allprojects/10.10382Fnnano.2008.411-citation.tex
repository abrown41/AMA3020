\section{\href{https://doi.org/10.1038/nnano.2008.411}{ Current-driven atomic waterwheels}}
 \subsection*{Supervisor: Dr Daniel Dundas}


A current induces forces on atoms inside the conductor that carries it. It is now possible to compute these forces from scratch, and to perform dynamical simulations of the atomic motion under current. One reason for this interest is that current can be a destructive force—it can cause atoms to migrate, resulting in damage and in the eventual failure of the conductor. But one can also ask, can current be made to do useful work on atoms? In particular, can an atomic-scale motor be driven by electrical current as it can be by other mechanisms For this to be possible, the current-induced forces on a suitable rotor must be non-conservative, so that net work can be done per revolution. Here we show that current-induced forces in atomic wires are not conservative and that they can be used, in principle, to drive an atomic-scale waterwheel.
