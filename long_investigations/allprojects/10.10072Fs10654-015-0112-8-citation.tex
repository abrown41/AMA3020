\section{\href{https://doi.org/10.1007/s10654-015-0112-8}{ Effect of major lifestyle risk factors, independent and jointly, on life expectancy with and without cardiovascular disease: results from the Consortium on Health and Ageing Network of Cohorts in Europe and the United States (CHANCES)}}
 \subsection*{Supervisor: Dr Karen Cairns}


Seldom have studies taken account of changes in lifestyle habits in the elderly, or investigated their impact on disease-free life expectancy (LE) and LE with cardiovascular disease (CVD). Using data on subjects aged 50+ years from three European cohorts (RCPH, ESTHER and Tromsø), we used multi-state Markov models to calculate the independent and joint effects of smoking, physical activity, obesity and alcohol consumption on LE with and without CVD. Men and women aged 50 years who have a favourable lifestyle (overweight but not obese, light/moderate drinker, non-smoker and participates in vigorous physical activity) lived between 7.4 (in Tromsø men) and 15.7 (in ESTHER women) years longer than those with an unfavourable lifestyle (overweight but not obese, light/moderate drinker, smoker and does not participate in physical activity). The greater part of the extra life years was in terms of “disease-free” years, though a healthy lifestyle was also associated with extra years lived after a CVD event. There are sizeable benefits to LE without CVD and also for survival after CVD onset when people favour a lifestyle characterized by salutary behaviours. Remaining a non-smoker yielded the greatest extra years in overall LE, when compared to the effects of routinely taking physical activity, being overweight but not obese, and drinking in moderation. The majority of the overall LE benefit is in disease free years. Therefore, it is important for policy makers and the public to know that prevention through maintaining a favourable lifestyle is “never too late”.
