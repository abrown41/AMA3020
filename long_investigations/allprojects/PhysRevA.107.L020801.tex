\section{\href{https://link.aps.org/doi/10.1103/PhysRevA.107.L020801}{ Unexpected dipole instabilities in small molecules after ultrafast XUV irradiation}}

\subsection*{Supervisor: Dr Daniel Dundas}



We investigate the depletion of deep-lying single-electron states in the N2 dimer under the influence of very short extreme-ultraviolet (XUV) pulses. We find, first, a marked occupation inversion for a certain window of XUV energies around 50 eV, where depletion of the deepest bound valence electron state is much larger than for any other state, and second, that this occupation inversion drives a dipole instability, i.e., a spontaneous reappearance of the dipole signal long after the laser pulse is over and the initial dipole oscillations have died out. As a tool for this study, we use time-dependent density functional theory with a self-interaction correction solved on a coordinate-space grid with absorbing boundary conditions. Key observables are state-specific electron emis- sion (depletion) and photoelectron spectra (PES). The dipole instability generates additional electron emission, leading to a specific low-energy structure in PES, a signal which could be used to identify the dipole instability experimentally. The here reported dedicated depletion of a deep lying electron state by a well-tuned XUV pulse has also been found in other atoms and molecules. It provides a practicable realization of an instantaneously produced deep hole state, a situation which is often assumed ad hoc in numerous theoretical studies of energetic ultrafast processes. Moreover, the identification of the subsequent dipole instability by PES will allow one to study basic decay channels of hole states in detail.
