\section{\href{https://doi.org/10.1103/PhysRevA.55.3760}{Multiphoton detachment of electrons from negative ions}}

\subsection*{Supervisor: Dr Gleb Gribakin}
A simple analytical solution for the problem of multiphoton detachment from negative ions by a linearly polarized laser field is found. It is valid in the wide range of intensities and frequencies of the field, from the perturbation theory to the tunneling regime, and is applicable to the excess-photon as well as near-threshold detachment. Practically, the formulas are valid when the number of photons is greater than one. They produce the total detachment rates, relative intensities of the excess-photon peaks, and photoelectron angular distributions for the hydrogen and halogen negative ions, in agreement with those obtained in other, more numerically involved calculations in both perturbative and nonperturbative regimes. Our approach explains the extreme sensitivity of the multiphoton detachment probability to the asymptotic behavior of the bound-state wave function. Rapid oscillations in the angular dependence of the n-photon detachment probability are shown to arise due to interference of the two classical trajectories, which lead to the same final state after the electron emerges at diametrically opposite sides of the atom when the field is close to maximal.
