\section{\href{https://pubs.acs.org/doi/10.1021/acs.iecr.8b05709}{Development and Optimization of an Immobilized Photocatalytic System within a Stacked Frame Photoreactor (SFPR) Using Light Distribution and Fluid Mixing Simulation Coupled with Experimental Validation}}

\subsection*{Supervisor: Dr Lorenzo Stella}
Recently, photocatalytic reactors have been designed with a view toward overcoming mass transfer limitations especially in systems with immobilized catalysts. This paper reports the design of a titanium “bladed” propeller with TiO2 immobilized on the blades. To evaluate the propeller efficiency, modeling using COMSOL Multiphysics was validated experimentally using coumarin as a probe molecule allowing for OH radical quantification. Modeling of light distribution and catalyst irradiance at varying irradiation distance was performed using ray optics, which, alongside experimental work, showed that irradiation at 4 and 5 cm from the propeller yielded the highest irradiance (29.3 and 22.1~mW/cm$^2$) and OH radical concentrations (5.38 and 5.56 $\mu$M, respectively). Propeller rotation was modeled and compared against experimental data to assess mass transfer limitations at varying rotation speeds. This showed that 300 rpm provided the highest rate of coumarin degradation (0.32 $\mu$M/min) despite the model showing higher fluid velocities at 400~rpm.
