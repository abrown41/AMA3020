\section{\href{https://doi.org/10.1016/j.ejor.2021.12.033}{A framework for developing generalisable discrete event simulation models of hospital emergency departments}}

\subsection*{Supervisor: Dr Laura Boyle}
Discrete event simulation (DES) is routinely used to model hospital emergency departments (EDs), primarily due to its ability to represent complex patient flow processes and investigate improvement strategies. Despite this, it is clear from published studies that many DES models are not subsequently implemented in hospitals or reused for other sites. This research addresses a gap in the literature by presenting a new data-driven modelling framework ‘GE-DES’, which outlines an approach to the design and development of generalisable ED models. The nature of the framework means that it is sufficiently flexible (i) for use across multiple EDs, and (ii) for investigating hospital-specific problems through data-driven customisation. The primary aim of GE-DES is to support model reuse and implementation. The framework is demonstrated through application to a case study ED in Australia.
