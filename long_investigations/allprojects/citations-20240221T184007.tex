\section{\href{https://doi.org/10.1093/mnras/sty3198}{ Towards converged electron-impact excitation calculations of low-lying transitions in Fe ii}}

\subsection*{Supervisor: Prof Cathy Ramsbottom}


Absorption and emission lines of the iron-peak species Fe ii are prominent in the infrared, optical, and ultraviolet spectra of a myriad of astrophysical sources, requiring extensive and highly reliable sets of atomic structure and collisional data for an accurate quantitative analysis. However, comparisons among existing calculations reveal large discrepancies in the effective collision strengths, often up to factors of three, highlighting the need for further steps towards new converged calculations. Here we report a new 20 configuration, 6069 level atomic structure model, calculated using the multiconfigurational Dirac-Fock method. Collision strengths and effective collision strengths are presented, for a wide range of temperatures of astrophysical relevance, from substantial 262 level and 716 level Dirac R-matrix calculations, plus a 716 level Breit-Pauli R-matrix calculation. Convergence of the scattering calculations is discussed, and results are critically compared with existing data in the literature, providing us with error estimates for our data. As a consequence, we assign an uncertainty of ±15 per cent to relevant forbidden and allowed transitions encompassed within a 50 level subset of the 716 level Dirac R-matrix data set. To illustrate the implications of our new data sets for the analysis of astronomical observations of Fe ii, they are incorporated into the cloudy modelling code, sample Fe ii spectra are generated and compared.
