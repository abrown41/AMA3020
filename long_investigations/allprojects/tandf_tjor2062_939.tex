\section{\href{https://doi.org/10.1057/jors.2010.27
}{ Using simulation to assess cardiac first-responder schemes exhibiting stochastic and spatial complexities
}}

\subsection*{Supervisor: Dr Karen Cairns}



A Monte-Carlo simulation-based model has been constructed to assess a public health scheme involving mobile-volunteer cardiac First-Responders. The scheme being assessed aims to improve survival of Sudden-Cardiac-Arrest (SCA) patients, through reducing the time until administration of life-saving defibrillation treatment, with volunteers being paged to respond to possible SCA incidents alongside the Emergency Medical Services. The need for a model, for example, to assess the impact of the scheme in different geographical regions, was apparent upon collection of observational trial data (given it exhibited stochastic and spatial complexities). The simulation-based model developed has been validated and then used to assess the scheme's benefits in an alternative rural region (not a part of the original trial). These illustrative results conclude that the scheme may not be the most efficient use of National Health Service resources in this geographical region, thus demonstrating the importance and usefulness of simulation modelling in aiding decision making.
