\section{\href{https://doi.org/10.1103/PhysRevA.93.023428}{High-order-harmonic generation in benzene with linearly and circularly polarised laser pulses}}

\subsection*{Supervisor: Dr Daniel Dundas}



High-order-harmonic generation in benzene is studied using a mixed quantum-classical approach in which the electrons are described using time-dependent density functional theory while the ions move classically. The interaction with both linearly and circularly polarised infra-red ($\lambda = 800$ nm) laser pulses of duration 10 cycles (26.7 fs) is considered. The effect of allowing the ions to move is investigated as is the effect of including self-interaction corrections to the exchange-correlation functional. Our results for circularly polarised pulses are compared with previous calculations in which the ions were kept fixed and self-interaction corrections were not included while our results for linearly polarised pulses are compared with both previous calculations and experiment. We find that even for the short duration pulses considered here, the ionic motion greatly influences the harmonic spectra. While ionization and ionic displacements are greatest when linearly polarised pulses are used, the response to circularly polarised pulses is almost comparable, in agreement with previous experimental results.
