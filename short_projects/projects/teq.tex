\section{Finding the truth in TEQs\\ \small{(%Pathway: SOR; 
Pre-requisite: None but willingness to code a bit)}}
%TEQ
One of the questions that students answer when filling in Teaching Evaluation
Questionnaires (TEQ) is about the percentage of lectures they attended. After
tallying, these results (in simplified form) are presented in a table like
this
\setlength{\extrarowheight}{3pt}
\begin{center}
\begin{tabular}{|l|c|c|c|c|}
\hline
\% of lectures attended & 25\% & 50\% & 75\% & 100\% \\
\hline
No. of answers & 7 & 19 & 42 & 29 \\
\hline
\end{tabular}
\end{center}
where the total number of answers is 97.\\[6pt]
Since the TEQs are filled in in a lecture, the frequencies in the table
above are \textit{biased}. Using the data from the table, estimate the true
frequencies that would be observed if the TEQ were filled by all 150 students
in the class. You can assume that attendances by individual students are
uncorrelated random variables. Using your result, check whether the attendance
of the lecture where TEQ was taken, was typical for the given class size.\\[6pt]
Generalise your approach to an arbitrary number of scores, and to a continuous
distribution.
In each case, determine the mean and standard deviation od the number of
students in class based on the true frequencies that you obtain from the
frequencies observed in the lecture.
