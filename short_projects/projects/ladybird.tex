\section{Ladybird lost\\ \small{(%Pathway: Any; 
Pre-requisite: Classical mechanics would help, but really none)}}
%Ladybird
In a round room of radius $R$, a large number of coins $N$ of diameter $d$ are
randomly dispersed upon the floor. A ladybird starts from the centre of the
room, crawling at speed $v$.
\begin{enumerate}\setlength{\itemsep}{0pt}
\item How long (on average) does it take before it meets a coin?
\item Suppose that every time the ladybird meets a coin, it changes direction at
random. How long (on average) before it makes it to the wall?
\item Suppose every time it `hits' a coin, the coin magically disappears.
Work out (approximately, and on average) the law of decrease of the number
of remaining coins as a function of time. (Assume that if, in the process,
the ladybird hits the wall, it is simply `reflected' back towards
the interior of the room, at a random angle.)
\item Calculate the answers for parts 1--3 for a room of typical size,
1p coins and $v = 1$~cm/s, and $N$ of your choice.
\end{enumerate}

{\it Keywords:} to work on this problem, you will need to think (or read)
about mean free paths, random walks/diffusion/Brownian motion.

%Extensions: one can ask many fun questions involving the above key words.
%Suggest one or two.
