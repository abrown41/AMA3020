\section{Paper-scissors-stone\dots\ and jelly babies!\\ \small{(%Pathway: Any, with a little bias to SOR; 
 Pre-requisite: None, but willingness to code)}}
%paper-scissors-stone
Four children play a long sequence of paper-scissors-stone games in
pairs. They divide up 40 jelly babies, starting with 10 each.\\[6pt]
The eldest child begins the sequence by choosing an opponent at
random, then the next eldest chooses at random, and so on, to complete the round
of 4 games. If a child wins a game, the loser gives the winner a jelly
baby. If the game is a draw no sweets are exchanged. In each game, there is
an equal probability of winning, losing or drawing.\\[6pt]
There are two versions of the rules:
\begin{itemize}\setlength{\itemsep}{0pt}
\item[(a)] If a child, at any time, has no sweets, they are eliminated.
\item[(b)] If a child with no sweets plays a child with sweets, they
automatically win.
\end{itemize}
Carry out a computer simulation of the game.\\[6pt]
For case (a) find out how long (how many rounds), on average,
the game lasts. How does the duration of the game vary with the
number of children and the number of sweets?\\[6pt]
For case (b) find out the fraction of time the oldest child has no
sweets, and the fraction of time when they have all the sweets. Can you
explain these values mathematically?
