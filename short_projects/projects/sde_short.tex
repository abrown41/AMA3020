\section{Random walks on Wall Street\\ \small{(%Pathway: Any with a small bias to AMA; 
Pre-requisite: None really, but you need to be willing to code and you should be familiar with differential equations)}}
%\item
%\underline{\large \bf Random walks on Wall Street} \\[0.1cm]
\setcounter{equation}{0}
The problem in predicting the prices of financial products is an apparently
random nature of their variation [1]. One mathematical model of price
fluctuations is the {\em random walk} [1,2], where one assumes that the price
jumps (up or down) are given by a normal distribution. This mathematical
problem is very closely related to the problem of Brownian motion solved
in 1905 by Einstein.\\[6pt]
Let $S(t)$ be the price of shares at time $t$. Let us assume that the change
in $S$ during time interval $dt$ can be described by the equation
\begin{equation}\label{eq:dS}
dS = a S  dt+ b S dW ,
\end{equation}
where $a$  and $b$ are constants. The second term in Eq. (\ref{eq:dS}) is a
random jump and $dW$ is a normally distributed random variable with
zero mean and variance $dt$. \\[6pt]
This project is about solving equation (\ref{eq:dS}) step-by-step using 
a computer, for one or two values of $a$ and $b$. This requires writing a
short program that generates the random numbers (jumps) so that you can
calculate $dS$. You can then compare your computed result for $S(t)$ with the
exact solution of the equation.\\[6pt]
%That is, if the initial value of $S$ is $S_0$ then after a time, $t$,
%$$
%S(t) = S_0 \exp \left[   (a-{\textstyle {1 \over 2}} b^2)t+b W(t) \right]
%$$
[1] J.~C. Hull J C,{\em Options, futures and other derivatives\/}.\\[3pt]
[2] P.~E. Kloeden, E. Platen, and H. Shurz, {\em Numerical solution of
stochastic differential equations through computer experiments\/}.
