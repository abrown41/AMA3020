\section{Double pendulum\\ \small{(%Pathway: AMA; 
Pre-requisite: Classical Mechanics)}}
% Double pentulum
A double pendulum is a pendulum with a second pendulum attached to it. We
assume that each pendulum consists of a mass $m$ at the end of a massless rod
of length $\ell$. Their positions are described by the angles $\phi _1$
and $\phi _2$ that each pendulum makes with the vertical.\\[6pt]
The Lagrangian of the system reads (verify!)
$$
L=\frac 12m\ell^2\left[2\dot\phi _1^2+\dot\phi _2^2+2\dot\phi _1\dot\phi _2
\cos(\phi _1-\phi _2)\right]+mg\ell(2\cos\phi _1+\cos\phi _2) 
$$

\begin{enumerate}\setlength{\itemsep}{0pt}
\item List the conserved quantities.
\item Describe qualitatively the motion of the pendulums with simple initial
conditions, e.g., one at rest and the other one with some initial velocity.
\item Write Lagrange's equations of motion for the variables $\phi _i$.
Investigate the motion for small $\phi _i$.
\item Use the Runge-Kutta [1] method to solve the equations of motion for
arbitrary initial conditions, using python or a programming language of your
choice.
\item Test your program with the initial conditions you discussed in 2 and
comment.
\end{enumerate}

[1] See for example: R. L. Burden and J. D. Faires, {\it Numerical Analysis}


