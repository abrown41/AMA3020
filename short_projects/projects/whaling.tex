The blue whale, the largest animal known to have lived on land or sea, was caught in large numbers
earlier this century and this resulted in a drastic decline in the size of the population.  
The blue whale is now officially a protected species.  However, it has been supposed that it will
become extinct in the near future because the number currently alive is so small that the males have a 
near impossible task of finding a mate. \\[0.2cm]
If the male and female populations, $M$ and $F$ respectively, are assumed, in the absence of encounters,
to decline at a rate proportional to the size of each population and the number of encounters is taken
to be proportional to $MF$, the equations determining $M$ and $F$ take the form
\begin{displaymath}
\frac{dM}{dt}~ = ~ - \alpha M ~ + ~ \beta MF
\end{displaymath}
\begin{displaymath}
\frac{dF}{dt}~ = ~ - \eta F ~ + ~ \gamma MF
\end{displaymath}
where $\alpha$, $\beta$, $\gamma$ and $\eta$ are positive constants. \\[0.2cm]
Analyse in detail the predictions this model makes about the behaviour of $M$ and $F$ with time.
Assign appropriate values to $\alpha$, $\beta$, $\gamma$ and $\eta$ and the values of $M$ abd $F$ at time
$t=0$, and use the equations to determine $M$ and $F$ at a sample range of subsequent times.  
Compare these results with any information you can obtain about the actual behaviour of the populations.
\\[0.2cm]
How would you modify the model to take into account
\begin{enumerate}
\item the effect of illegal whaling;
\item the limitation imposed on the population by the finite resources available to maintain the whales?
\end{enumerate}
