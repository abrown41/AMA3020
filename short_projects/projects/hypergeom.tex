\section{Hypergeometric functions in Quantum Theory\\ \small{(%Pathway: Any; 
Pre-requisite: Quantum theory)}}


The confluent hypergeometric function $_1F_1$ and the Gauss hypergeometric function $_2F_1$ are defined by:
\begin{align}
_1F_1(a,c;z) &= \sum_{n=0}^{\infty} \frac{(a)_n z^n}{(c)_n n!} \\
_2F_1(a,b,c;z) &= \sum_{n=0}^{\infty} \frac{(a)_n (b)_n z^n}{(c)_n n!}
\end{align}
where the Pochhammer symbol $(a)_n$ represents the rising factorial:
\begin{displaymath}
(a)_n = \begin{cases}
a(a+1)(a+2) \cdots (a+n-1) & (n \geq 1) \\
1 & (n=0)
\end{cases}
\end{displaymath}

Show that Hermite polynomials $H_n(x)$, Laguerre polynomials $L_n(x)$ (or
associated Laguerre $L_n^{\alpha}(x)$), and Legendre polynomials $P_n(x)$ can
each be written as hypergeometric functions with specific parameter choices.
Show that for certain quantum system(s) like
\begin{itemize}
   \item Quantum harmonic oscillator (Hermite)
   \item Hydrogen atom radial equation (Laguerre)
   \item Angular momentum/Legendre equation (Legendre)
\end{itemize}
the differential equation leads to a hypergeometric series solution.

{\bf References}\\[3pt]
Abramowitz \& Stegun, Handbook of Mathematical Functions (1970), Chapters 13, 15, 22

