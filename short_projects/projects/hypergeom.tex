
\item
\underline{\large\bf Hypergeometric functions in Quantum Theory}  \\[0.1cm]
In Quantum Theory, there arise Hermite, Laguerre and Gauss polynomials.  They are best understood in 
the framework of the hypergeometric function $_pF_q$. \\[0.2cm]
Typically, we have
\begin{displaymath}
_1F_1(a,c;z) ~ \equiv ~ \sum_{n=0}^{\infty} \frac{(a)_n z^n}{(c)_n n!}
\end{displaymath}
and
\begin{displaymath}
_2F_1(a,b,c;z) ~ \equiv ~ \sum_{n=0}^{\infty} \frac{(a)_n (b)_n z^n}{(c)_n n!}
\end{displaymath}
where the Pochhammer symbol is defined by
\begin{displaymath}
(a)_n = \left \{ \begin{array}{cl}
a(a+1)(a+2) \cdots (a+n-1) & (n \geq 1) \\
1 & (n=0)  \end{array} 
\right .
\end{displaymath}
The special functions of quantum mechanics should be systematically interpreted using this notation.  
This facilitates the discovery of the eigenvalues, as a result of terminating series in contrast to divergent 
infinite series. \\[0.5cm]
(ref: M.\ Abramowitz and I.A.\ Stegun, (1970): {\em Handbook of Mathematical Functions}) \\

