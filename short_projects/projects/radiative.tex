\section{Radiative transfer in stars\\ \small{(%Pathway: Any; 
Pre-requisite: None)}}

The transfer of radiation through a star can be written
as the differential equation
\begin{eqnarray}
\frac{1}{\kappa_\lambda}\frac{{\rm d}I_\lambda}{{\rm d}s_\lambda}=
 -I_{\lambda}+S_\lambda
\nonumber
\end{eqnarray}
where $I_\lambda$ = radiation intensity at wavelength $\lambda$,
$\kappa_\lambda$ = absorption coefficient at $\lambda$, \linebreak[4]
${\rm d}s$ = path length,
$S_\lambda$ = source function.
Investigate the solution in the optically thin ($\tau_\lambda\ll 1$)
and optically thick ($\tau_\lambda\gg 1$)
limiting cases, with zero and non-zero starting values of $I_\lambda$.
\\[0.3cm]
\underline{Suggested reading}
\\
You may want to consult a text book on astrophysics for
more detailed background to this equation, eg. 
B\"{o}hm-Vitense, `Introduction to stellar astrophysics' vol. 2 (CUP, 1989). \\
