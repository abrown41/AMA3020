\section{Time and tides\\ \small{(%Pathway: Any; 
Pre-requisite: None)}}
%\item
%\underline{\large\bf Time and Tides}  
The times of low and high tides are important for holidaymakers as well as
fishermen at sea. Tidal information can also be used to spot
tsunamis well in advance of them reaching shore. This project investigates
the mathematics behind so-called tide tables.\\[6pt]
The height of the tide $h(t)$ at any point on the Earth depends
on many factors including its geographical location, coastline, ocean
currents and storms. The primary cause of the tides is the force of gravity
due to the Moon and the Sun, that combine to give the daily, monthly and
seasonal tides. The three main cycles are: the daily rotation of the Earth
$\omega_1$, the monthly rotation of the moon around the Earth $\omega_2$, and
the annual motion of the Earth around the sun $\omega_3$.
There are also two small lower-frequency corrections for the precession of
the Earth-Moon orbit (the perigee and orbital plane) but these can be
neglected.\\[6pt]
The simplest model of the tide height is given by
\[
h(t) = h_0 + a \cos\Omega t + b \sin\Omega t,
\]
with $h_0$ the average sea level. The frequency $\Omega $ has a well-known
value [1]
\[
\Omega = 2(\omega_1 - \omega_2 + \omega_3) = 28.984~\mbox{deg/hr},
\]
while the remaining parameters $h_0$, $a$ and $b$ depend on the location of the
port [2]. This investigation requires you to find them. This can be done by
using data for $h(t)$ from tide tables for a few days, and solving the linear
equations to determine $h_0$, $a$ and $b$. Once these are found, you can
{\em predict} the times for high and low tides over the whole week or
fortnight. Do this for Bristol and Donaghadee and then compare your forecast
with the forecasts or measurements made at these harbours. Can you suggest
any improvements to your model, and how these might be implemented?\\[6pt]
You may also use data from the NOAA Tsunami center [3].\\[6pt]
[1] \verb|www.math.sunysb.edu/~tony/tides/harmonic.html|\\[3pt]
[2] \verb|http://news.bbc.co.uk/weather/coast_and_sea/tide_tables/|\\[3pt]
[3] \verb|www.ndbc.noaa.gov/dart.shtml|
