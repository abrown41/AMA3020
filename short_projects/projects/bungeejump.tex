\section{Bungee jumping\\ \small{(%Pathway: Any; 
Pre-requisite: Classical Mechanics)}}
%\item
%\underline{\large \bf Bungee jumping}\\[0.1cm]
In bungee jumping, a person jumps from a tall structure, e.g., a bridge,
while attached to a long, heavy elastic band. The elastic band ensures that the
person does not hit the surface below. The aim of this project is to find
the acceleration the person experiences while falling.\\[6pt]
To simplify the problem, you can replace the elastic band with a (non-elastic)
rope. Assume also that the rope initially hangs down from the structure
and the person.
\begin{itemize}\setlength{\itemsep}{0pt}
\item How does the speed of the person depend on the distance travelled?
Determine the acceleration of the person by taking the derivative of the
velocity with respect to time. How does it compare to $g$? [Hint: use energy
conservation.] 
\item What happens for other initial positions of the rope?
\item Extend, for example, by numerically determining the full motion, or by
replacing the rope by an elastic band.
\end{itemize}
