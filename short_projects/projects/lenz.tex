\section{Lenz's potentials\\ \small{(%Pathway: AMA; 
Pre-requisite: Classical Mechanics)}}
% Lenz's potentials 
For a particle moving in a central field $U(r)$, the path can be found using
plane polar coordinates $r$ and $\phi $ in terms of an indefinite integral
[1]. However, there are only a handful of potentials for which this integral
and the equation of the path can be found analytically. You are familiar with
one example, namely the gravitational potential $U(r)=-\alpha /r$, in which
the paths are conic sections (ellipse, parabola or hyperbola). Using this
as a starting point, investigate the paths of the particle with zero energy
($E=0$) in {\it Lenz's potential}
\[
U(r)=-\frac{2uR^2}{r^2\left[\Bigl(\dfrac{r}{R}\Bigr)^\mu +
\Bigl(\dfrac{R}{r}\Bigr)^\mu \right]^2},
\]
where $u$, $R$ and $\mu $ are constants, for different values of $\mu $.
Such potentials find some unexpected applications in atoms and clusters
[V. N. Ostrovsky, Phys. Rev. A {\bf 56}, 626 (1997)].\\[6pt]
[1] L. D. Landau and E. M. Lifshitz, \textit{Mechanics}
(Butterworth-Heinemann, Oxford, 2001).

