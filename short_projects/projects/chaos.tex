\section{Chaos and nonlinearity\\ \small{(%Pathway: Any with a small bias to AMA; 
Pre-requisite: None)}}
%\item
%\underline{\large \bf Chaos and non-linearity} \\[0.1cm]
One of the most interesting examples of chaos is given by the logistic equation
$$
x_n=rx_{n-1}(1-x_{n-1}),
$$
where, choosing a value $0 < x_0 < 1$ and a fixed number $0 \leq r \leq 4$, we
can calculate a sequence of values $\{x_1, x_2, ..., x_N\}$ by repeated use of
this equation. As $r$ increases, the regular behaviour changes to chaos.
\\[6pt]
By investigating the behaviour as a function of $r$, find out when
bifurcations start, and when chaos starts. Can you expand on your findings?
\\[6pt]
{\bf References}\\[3pt]
J. Gleick, {\it Chaos: making a new science\/}.\\[3pt]
S.~H. Strogatz, {\it Nonlinear dynamics and chaos\/}.
