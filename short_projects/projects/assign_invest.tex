\section{Assigning investigations\\ \small{(%Pathway: Any with a little bias for SOR; 
Pre-requisite: None)}}
%Assigning Investigations

In order to be assigned an investigation project from a list, students
in a module are asked for their 5 choices without particular preferences.
You are faced with the task of assigning each student one of the projects of
their choice in a way that no two students are assigned the same topic
(\textit{satisfiability problem}). Assume that the number of projects $K$ is
greater than or equal to the number of students $N$. Devise an algorithm based
on backtracking or other forms of search that finds a feasible solution to this
problem. Discuss examples.\\[6pt]
Assume now that the 5 students' choices are ordered from their best
preference to the least one. In this case use the Hungarian algorithm, or
any other of your choice, to find the assignment that maximises the overall
students' preferences. Discuss the adequacy of 5 preferences for assigning
projects and the relationship between the number of students and number of
projects available. (Note that some projects can be much more popular
than others).
