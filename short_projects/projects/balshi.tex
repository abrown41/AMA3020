\section{Weighing on a ship\\ \small{(%Pathway: AMA;
 Pre-requisite: Classical Mechanics)}}
%\item
%\underline{\large \bf Weighing on a ship}\\[0.1cm]
The simplest way of determining the mass of an object is by measuring the
gravitional force $mg$ on this object. This is fine on land as the Earth is
(approximately) an inertial frame of reference and $g\approx \mbox{const}$.
However, a ship rocked by waves is not an inertial frame, and the apparent
``gravity'' is not constant. This needs to be taken into account when, for
example, weighing the catch. In this project you will investigate how this
can be done.\\[6pt]
G. Kessling, D. Birnbacher and C. Berg, {\it Meas. Sci. Technol.\/} {\bf 4},
1035--1042 (1993).
