\section{A model for giving up smoking\\ \small{(%Pathway: Any; 
Pre-requisite: Familiarity with differential equations)}}
%\item
%\underline{\large\bf A model for giving up smoking} \\[0.1cm]
Let $S(t)$ denote the number of smokers at time $t$, $P(t)$ the number of
potential smokers, i.e., those who have not taken it up yet, and $Q(t)$ the
number of smokers who have quit permanently, and let $N=S(t)+P(t)+Q(t)$ be
the total population size which is assumed to be constant.
A possible model of the variation of $S(t)$, $P(t)$, and $Q(t)$ with time is given by
\begin{eqnarray*}
\frac{dP(t)}{dt} &\!\!\! =\!\!\! & \mu N - \beta P(t) \frac{S(t)}{N} - \mu P(t),\\
\frac{dS(t)}{dt} &\!\!\! =\!\!\! & \beta P(t) \frac{S(t)}{N} - (\mu + \gamma) S(t),\\
\frac{dQ(t)}{dt} &\!\!\! =\!\!\! & \gamma S(t) - \mu Q(t),
\end{eqnarray*}
where $\beta$ determines the rate at which potential smokers take up
smoking due to contact with smokers, $1/\gamma $ is the average time as a
smoker, and $1/\mu$ is the average time in the population.
\begin{itemize}\setlength{\itemsep}{0pt}
\item Explain each term in the above differential equations. What assumptions
are being made in this model?
\item Re-write the equations in terms of the population fraction variables
\begin{displaymath}
x(t) = \frac{P(t)}{N}, \quad y(t) = \frac{S(t)}{N}, \quad z(t) = \frac{Q(t)}{N},
\end{displaymath}
and show by changing the time variable to $\tau =\mu t$, that the behaviour
of the system depends only on $\beta /\mu $ and $\gamma /\mu $. 
\item Study the equilibrium positions of this system of equations, and the
behaviour of $x,y,z$ close to equilibrium positions.
\item Investigate simple modifications to the model.
\end{itemize}
