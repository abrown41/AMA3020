\section{``Impossible'' exam question\\ \small{(%Pathway: Any; 
Pre-requisite: None)}}
% Impossible economics exam
% Insert 
% \usepackage[colorlinks,urlcolor=blue]{hyperref}
% in the preamble of the main file


In January 2015, some final year economics students at Sheffield University
claimed that their exam contained an ``impossible question'' [1]. However,
the question, which is reproduced below, should give no trouble
to a mathematics student! Moreover, you should also be able to find ways of
extending it.

% Include the question exactly as it appeared on that exam, as a picture
%
%\includegraphics*[angle=0,width=15cm]{sheffieldeconomics.jpg}
%
% or use the text below.

\vspace{12pt}

{

Consider a country with many cities and assume that there are $N>0$ people in
each city. Output per person is $\sigma N^{0.5}$ and there is a coordination
cost per person of $\gamma N^2$. Assume that $\sigma >0$ and $\gamma >0$.

\begin{itemize}
\item[(a)] What sort of things does the coordination cost term $\gamma N^2$
represent? Why does it make sense that the exponent on $N$ is greater than 1?
\hfill[10 marks]

\item[(b)] Draw a graph of per-capita consumption as a function of $N$ and
derive the optimal city size $N$. How does it depend on the parameters $\sigma $
and $\gamma $? Provide intuition for your answers.\hfill[10 marks]

\item[(c)] Describe which combinations of $\sigma $ and $\gamma $ generate a
peasant economy, meaning an economy with no cities (or 1-person cities). Why
might the values of the parameters $\sigma $ and $\gamma $ have changed over
time? What do these changes imply in terms of the optimal city size?
\hfill[10 marks]

\end{itemize}
}

[1] \href{http://www.bbc.com/news/education-31057005}{http://www.bbc.com/news/education-31057005}.
