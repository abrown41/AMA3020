\section{Electrons in a laser field\\ \small{(%Pathway: AMA; 
Pre-requisite: Electromagnetism would help but is not necessary. Familiarity with coding)}}
%\item
%\underline{\large\bf Electronic motion in a light field}\\[0.1cm]
When an atom is irradiated by a strong laser pulse, its electric
field can detach one of the electrons. What happens
next is determined by the motion of this electron in the laser field.\\[6pt]
Consider an electron that emerges {\em with zero velocity} from an atom at the
origin ($x=0$) at $t=0$ and is acted upon by a periodic electric force in
the $x$ direction,
$$F(t)=eE_0\cos (\omega t+\phi ),$$
where $e$ is the elementary charge, $E_0$ is the amplitude of the electric
field, $\omega $ is its frequency, and $\phi $ is the initial phase.
\begin{itemize}\setlength{\itemsep}{0pt}
\item By solving Newton's equation, determine the subsequent
motion of the electron.
\item Driven by the field, the electron can be directed back towards
its parent atom. Investigate the motion of the electron for different phases
$\phi $ and determine whether it returns to the origin.
If it does, find the kinetic energy of the electron upon its return, as a
function of the phase $\phi $ (e.g., for $\phi $ between $-\pi /2$
and $\pi /2$). What is the maximum value of this
energy? [Note that this problem may only have a numerical solution and python
can be used to find it.]
\item In a real experiment, He atoms are irradiated with laser light with
wavelength $\lambda =800$~nm [Th. Weber {\em et al.}, Phys. Rev. Lett.
{\bf 84}, 443 (2000)]. Determine the intensity of the laser light that
will allow the returning electron to knock out the second electron from
the same atom.
\end{itemize}
