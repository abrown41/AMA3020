\underline{\bf The Dirac delta function and entropy of harmonic oscillators} \hfill {\em Dr.\ Alavi} 
\\[0.1cm]
The Dirac delta function $\delta(x)$ appears in many branches of mathematical physics.  It has a number of
formal properties, including
\begin{eqnarray}
\delta(x) & = & 0 \hspace*{0.5cm} (x \neq 0) \\[0.2cm]
\int_{- \infty}^{\infty} \delta(x) ~  dx & = & 1 \\[0.2cm]
\int_{- \infty}^{\infty} \delta(x-x_0) ~ f(x)~ dx & = & f(x_0) \\[0.2cm]
\delta(f(x)) & = & \left| \frac{df(x)}{dx} \right| ^{-1} \delta(x-x_0) 
\end{eqnarray}
where in the last formula $x_0$ is chosen so that $f(x_0)$=0.  We are going to use the Dirac delta function
to evaluate the entropy of a system of harmonic oscillators. \\[0.2cm]
Consider an $N$ dimensional harmonic oscillator with the following Hamiltonian
\begin{displaymath}
H ~ = ~ \sum_{i=1}^N \frac{p_i^2}{2} ~ + ~ \sum_{i=1}^N \frac{x_i^2}{2}
\end{displaymath}
where $\{x_i\}$ are the coordinate variables and $\{p_i\}$ are the corresponding momenta.
\\[0.2cm]
If the system has total energy $E$, then the entropy $S$ of the system is given by
\begin{displaymath}
e^S ~ = ~ \int \delta(H-E)~dx_1dx_2...dx_Ndp_1dp_2...dp_N
\end{displaymath}
By investigating the properties of the Dirac delta function, obtain analytic formulae for $S$ as a 
function of $E$ and $N$. Show that
\begin{displaymath}
S ~ = ~ C(N) ~ + ~ (N-1) \log E
\end{displaymath}
where $C(N)$ is a constant dependent only on $N$.  Hence show that the entropy of the one-dimensional 
oscillator is independent of $E$. \\[0.2cm]
[Hint: Consider the special cases with $N$=1,2,3 respectively, before attempting the general case.]
