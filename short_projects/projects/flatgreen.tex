\section{Flat green bowling\\ \small{(%Pathway: AMA; 
Pre-requisite: Classical Mechanics)}}
%\item
%\underline{\large\bf Flat green bowling} \\[0.1cm]
This is a game so called because it involves rolling balls on a flat
horizontal surface, either made of grass or (when played indoors) consisting
of long mats. The aim of the game is for one's bowl(s) to reach as near as
possible to a specified position (the position of the `jack').\\[6pt]
It is made more interesting by the bowls having their mass distributed
non-spherically -- they have one axis of rotational symmetry with the centre
of mass on this axis but not at the geometrical centre of the bowl.
(In practice, the bowls are not quite geometrically spherical either, but this
can be thought of as a `second-order' effect.) This lack of symmetry results
in the path of the bowl being curved rather the straight, as it would be for
a spherically symmetric bowl.\\[6pt]
Find the path of the bowl, for a given initial speed and eccentricity of the
centre of mass.
