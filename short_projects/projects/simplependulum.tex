\section{The simple pendulum\\ \small{(%Pathway: AMA; 
Pre-requisite: Classical Mechanics)}}
%\item
%\underline{\large\bf The simple pendulum} \\[0.1cm]
The bob of a simple pendulum of length $l$ moves in accordance with the
differential equation 
\begin{displaymath}
\ddot{\theta} + \frac{g}{l} \sin \theta = 0,
\end{displaymath}
where $\theta$ is the angle the pendulum makes with the vertical and $g$ is
the acceleration due to gravity. The most elementary approach in classical mechanics books considers only small
oscillations for which  $|\theta | \ll 1$ (in radians, of course!). In this
case $\sin \theta \simeq \theta $ and the equation describes simple harmonic
motion,
\begin{displaymath}
\theta (t)= a \cos (\omega t + \alpha ),
\end{displaymath}
with the amplitude $a$, initial phase $\alpha $, angular frequency
$\omega = \sqrt{g/l}$ and period $T=2\pi /\omega $.\\[6pt]
What happens if $\theta$ is not small?
