 \section{Monte Carlo integration\\ \small{(%Pathway: Any; 
 Pre-requisite: None)}}
%\item
%\underline{\large \bf Monte Carlo integration}
Integrals can not always be found analytically in a closed form. Numerical
methods like Gauss-Hermite integration, allow one to calculate
single- and multidimensional integrals in a few dimensions (2, 3, perhaps 4)
by evaluating the integrand for a number of points in the integration
domain. These methods, however, scale as $M^N$ where $M$ is the number of grid
points in each dimension and $N$ is the dimensionality.\\[6pt]
An alternative methodology for calculating integrals in a large number
of dimensions is based on {\em stochastic sampling} of the integration domain.
Instead of evaluating the function on a regular grid of points, these points
are chosen randomly, but in such a way that the most relevant regions of
the integration domain are sampled more accurately (importance sampling).
This is called Monte Carlo integration.\\[6pt]
The project requires one to
\begin{itemize}\setlength{\itemsep}{0pt}
\item [(a)]Describe the method of Monte Carlo multidimensional integration,
including the idea of importance sampling and the Metropolis algorithm.
\item [(a)]Write a program for Monte Carlo (using python or any other
programming language), and apply it to a standard test problem.
Are there any approaches to improve the accuracy of the calculation?\\[6pt]
\end{itemize}
{\bf References}\\[3pt]
M. Allen and D. Tildesley, {\it Computer Simulation of Liquids\/}.\\[3pt]
D. Frenkel and B. Smit, {\it Understanding Molecular Simulation: from
Algorithms to Applications\/}.\\[3pt]
K. Binder, editor, {\it Monte Carlo Methods in Statistical Physics\/}.
