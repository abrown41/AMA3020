 \section{The Jaynes-Cummings model\\ \small{(%Pathways: AMA; 
 Pre-requisite: Quantum Theory)}}
%\item
%\underline{\large\bf The Jaynes-Cummings model}  \\[0.1cm]
Only very few models in quantum mechanics can be solved analytically. These
models are of great interest as they allow the most detailed
investigation of what is actually happening in quantum systems. The
Jaynes-Cummings model is one such model. It provides a fundamental
basis for understanding of the physics behind quantum computing.
\\[6pt]
The Jaynes-Cummings model considers a two-level atom interacting with laser
light, whose frequency is in resonance with the transition energy between the
levels.
Under these assumptions we can describe the state of the atom completely by a
two-dimensional (complex) vector of unit length. The interaction with the
laser field is then given by a $2\times 2$ complex matrix. This interaction
contains two parts: if the atom is in the ground state, it can absorb a photon
and reach the excited state, while if the atom is in the excited state, it can
emit a photon and drop back into the ground state. This is
described by the following interaction Hamiltonian
\[
H = \lambda \hbar \left(a\sigma_+ +a^{\dag} \sigma_-\right),
\]
where $\lambda$ is the interaction strength, $\hbar$ is Planck's constant,
$a$ and $a^\dagger$ are the photon annihilation (absorption) and creation
(emission) operators and $\sigma_+$ and $\sigma_-$ the atomic raising and
lowering operators. (The two states of the atom can be viewed mathematically
as ``spin-down'' and ``spin-up'' states, hence the use of Pauli
matrices, $\sigma _{\pm }=\sigma _x\pm i\sigma _y$.)\\[6pt]
In the project, the aim is to understand the Jaynes-Cummings model, and to
obtain the time evolution operator $U(t)$ for the interaction  Hamiltonian,
where
\[
U(t) = e^{-iHt}.
\]
{\bf References}\\[3pt]
S.~M. Barnett and P.~M. Radmore, {\em Methods in Theoretical Quantum Optics},
Ch.~2.\\[3pt]
W. P. Schleich, {\it Quantum Optics in Phase Space}.

