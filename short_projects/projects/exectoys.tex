\section{Impacts and executive toys\\ \small{(%Pathway: AMA; 
Pre-requisite: Classical Mechanics)}}
%\item
%\underline{\large\bf Impacts and executive toys} \\[0.1cm]
There are various `executive toys' involving the collisions of spheres --
Newton's cradle is a typical example. Another can be developed as follows. 
\begin{itemize}\setlength{\itemsep}{0pt}
\item Two spheres $A$ and $B$ of unequal mass can move freely on a thin
horizontal wire threaded through their centres. There is a buffer at one end
of the wire, and both $A$ and $B$ are projected with speed $V$ towards the
buffer, with $A$ closer to the buffer than $B$. Let the coefficient of
restitution for all impacts be $e$.  After $A$ and $B$ collide, sphere $A$ is
reduced to rest. Determine the ratio of the masses of the two spheres for
this effect to occur, and the speed which sphere $B$ attains after its
collision with $A$.
\item
Suppose there are $n$ spheres, each initially projected with the same speed
$V$ towards the buffer, and with a sufficient gap between them that collisions
take place in order. If all but the final sphere is reduced to rest by this
sequence of impacts, determine its mass as a proportion of the sum of the
masses of all the spheres, and its final speed. (You might find it helpful to
work with $e$=1 first and then generalise.)
\item
Suppose that the wire is now mounted vertically and the buffer is at its
bottom end. Discuss this case when the `executive toy' consists of three
spheres.
\end{itemize}
