\section{Harmonic oscillators\\ \small{(%Pathway: Any with a small bias to AMA; 
Pre-requisite: None really, but Classical Mechanics would make things easier)}}
%\item
%\underline{\large\bf Harmonic oscillators}  \\[0.1cm]
A point $P$ moves so that its position vector ${\bf r}$ relative to
the origin $O$ satisfies the equation  $\ddot {\bf r}=-\omega^2{\bf r}$, where
$\omega $ is a constant. Show that the motion of $P$ is compounded from two
harmonic oscillations executed with the same frequency at right angles to each
other and that the locus  of $P$ is an ellipse with the centre at $O$.\\[6pt]
Show further that if $P$ has Cartesian coordinates $(x,y)$ in this plane and
$O$ is the origin of coordinates, the locus of $P$ has the equation
\begin{displaymath}
\frac{x^2}{a^{2}_{1}}+\frac{y^2}{a^{2}_{2}}-\frac{2xy}{a_1a_2}
\cos(\epsilon_2-\epsilon_1)=\sin^2(\epsilon_2-\epsilon_1),
\end{displaymath}
where $a_1$ and $a_2$ are the amplitudes of the harmonic oscillations and
$\epsilon_1$ and $\epsilon_2$ are their respective phases.\\[6pt]
Discuss the shape and orientation of this locus for different values of
$\epsilon_2-\epsilon_1$.\\[6pt]
If the two harmonic oscillations are now taken to have different
frequencies, obtain the equation of the path of $P$.
