\section{Two coupled spins\\ \small{(%Pathway: AMA; 
Pre-requisite: Quantum Theory)}}
%\item
%\underline{\large\bf Molecular energy levels}  \\[0.1cm]
\setcounter{equation}{0}
An NMR experiment on a molecule containing two nuclei $A$ and $B$, is described
by the spin Hamiltonian
\begin{eqnarray}
H&\!\!\!=\!\!\!& (\omega_0+ \delta/2) \hat I_{Az}+ (\omega_0- \delta/2)
\hat I_{Bz} + J\, \hat{\bf I}_A\cdot \hat {\bf I}_B \nonumber \\
&\!\!\!=\!\!\!&(\omega_0+ \delta/2) \hat I_{Az}+
(\omega_0- \delta/2) \hat I_{Bz} + J\left[ \hat I_{Az}\hat I_{Bz}+
{1 \over 2}(\hat I_{A+}\hat I_{B-}+\hat I_{A-}\hat  I_{B+})\right],
\label{eq:Hspin}
\end{eqnarray}
where $\hat{\bf I}_A$ and $\hat{\bf I}_B$ are the nuclear spin operators
with components $\hat I_{Az}$ and $\hat I_{A\pm }=\hat I_{Ax}\pm i\hat I_{Ay}$
(and the same for $B$), and $\omega _0$, $\delta $ and $J$ are constants.
The first two terms describe the interaction of the spins with a magnetic
field in the $z$ direction, and the third term decribes the interaction between
the spins.\\[6pt]
For spin-$\frac{1}{2}$ nuclei the spin operators for each of the nuclei
are Pauli matrices, and for $J=0$ the system can be found in one of the
four states (``up-up'', ``up-down'', ``down-up'' and ``down-down''). Using
these as basis states, the eigenvalues of the Hamiltonian (\ref{eq:Hspin})
can be found by diagonalising a $4\times 4$ matrix.\\[6pt]
Hence, set up this matrix and find the spectrum of eigenvalues for 
$\omega_0=1000$, $\delta=10$, and $J=0,~5,~10,~20,~100$.\\[6pt]
See {\em Nuclear Magnetic Resonance Spectroscopy}
by R. K. Harris for background reading.

%(refs: Lynden-Bell and Harris Nuclear Magnetic Resonance Spectroscopy
%Chap 2; R.K.Harris  Nuclear Magnetic Resonance Spectroscopy Chap 2)

